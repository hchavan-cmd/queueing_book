% arara: pdflatex: { shell: yes, interaction: nonstopmode }
% arara: pythontex: {verbose: yes, rerun: modified }
% arara: pdflatex: { shell: yes, interaction: nonstopmode }
% arara: clean: { extensions: [ aux, blg, idx, ilg, ind, log, out, pytxcode, rel, toc ] }
% !arara: clean: { files: [ ans.tex, hint.tex] }

% arara: pdflatex
% arara: clean: { extensions: [ aux, blg, idx, ilg, ind, log, out, pytxcode, rel, toc ] }
% !arara: clean: { files: [ ans.tex, hint.tex] }


\documentclass[queueing-book-solution.tex]{subfiles}
%\externaldocument{queueing-book}

\opt{solutionfiles,check}{
\loadgeometry{tufte}
\Opensolutionfile{hint}
\Opensolutionfile{ans}
}

\begin{document}


\section{$G/G/1$ queues in Tandem}
\label{sec:tandem-queues}


Consider two $G/G/1$ stations in tandem.\sidenote{`In tandem' means `in line', one station after the other.}
Suppose we have the financial means to reduce the variability of the processing times at  one of the stations, but not at both.
Then we like to improve the one that has the most impact on the total waiting time in the line.

For the waiting time of the first machine, we can use Sakasegawa's formula, but to apply this to the second machine, we need $C^2_{a,2}$, i.e., the SCV of the inter-arrival times at the second station.
Now, noting that the output of the first machine forms the input of the second machine, it is clear that $C^2_{a,2}=C^2_{d,1}$, where $C^2_{d,1}$ is the SCV of the \emph{inter-departure} times of \emph{first} station.

In this section, we  present a formula\sidenote{The literature provides algorithms to deal with networks of $G/G/1$ queues in which the output of several stations form the input of another station and rework is allowed.
  In~\cref{sec:jackson-networks} we analyze such networks, but only for  $M/M/c$ stations.}
 to approximate $C^2_{d,1}$, and with this, we can model the propagation of variability through a tandem network of $G/G/c$ stations.



Let us consider the inter-departure times of a $G/G/1$ queue.
Suppose that the  utilization $\rho$ is very high.
Then the server will seldom be idle, so that most of the inter-departure times are the same as service times.
If, however, the utilization is low, the server will be idle most of the time, and the inter-departure times must be approximately equal to the inter-arrival times.
We obtain an  approximation for the SCV $C_{d}^2$ of the inter-departure times by  interpolating between these two extremes\sidenote{For the $G/G/c$ 
there is the generalization $
 C_{d}^2 \approx 1 + (1-\rho^2)(C_{a}^2-1) + \frac{\rho^2}{\sqrt{c}}(C_{s}^2-1)$. It is simple to see that this reduces to~\cref{eq:40}.}: 
\begin{equation} \label{eq:40}
 C_{d}^2 \approx  (1-\rho^2) C_{a}^2 + \rho^2 C_{s}^2.
\end{equation}

Combining Sakasegawa's formula with this expression provides us with a  very useful insight for a line. 
If we reduce $C^2_{s,1}$, i.e., the SCV of service times at the first station,   $\E{\W_1}$ and $C^2_{d,1}$ become smaller. Since  $C^2_{a,2}=C^2_{d,1}$, $\E{\W_2}$ becomes lower too, but also $C^2_{d,2}$, and so on. 
In other words, the entire chain benefits from an improvement in service variability at  the first station.\sidenote{Try to improve at the start.}



\begin{exercise}\label{ex:l-127}
Consider two $G/G/1$ stations in tandem.
Suppose $\lambda=2$ per hour, $C_{a,1}^2=2$, $C_{s,1}^2=C_{s,2}^2 = 0.5$, and $\E{S_1}=20$ minutes and $\E{S_2}=25$ minutes.
What is $\E\J = \E{J_1}+ \E{J_2}$? 
\begin{solution}
First station 1.
\begin{pyconsole}
labda = 2.0
S1 = 20.0 / 60
rho1 = labda * S1
ca1 = 2.0
cs1 = 0.5
EW1 = (ca1 + cs1) / 2 * rho1 / (1 - rho1) * S1
EJ1 = EW1 + S1
EJ1
\end{pyconsole}

Now station 2. We first need to compute $C_{d1}^2$. 

\begin{pyconsole}
cd1 = (1 - rho1 ** 2) * ca1 + rho1 ** 2 * cs1
cd1
labda = 2
S2 = 25.0 / 60
rho2 = labda * S2
ca2 = cd1  # here we use our formula
cs2 = 0.5
EW2 = (ca2 + cs2) / 2 * rho2 / (1 - rho2) * S2
EJ2 = EW2 + S2
EJ2
EJ1 + EJ2
\end{pyconsole}

\end{solution}
\end{exercise}


\opt{solutionfiles}{\Closesolutionfile{hint}
\Closesolutionfile{ans}
\loadgeometry{normal}
\input{hint}
\input{ans}
}

\end{document}



%%% Local Variables:
%%% mode: latex
%%% TeX-master: t
%%% End:
