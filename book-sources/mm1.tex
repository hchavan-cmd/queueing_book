% arara: pdflatex: { shell: yes, interaction: nonstopmode }
% arara: pythontex: {verbose: yes, rerun: modified }
% arara: pdflatex: { shell: yes, interaction: nonstopmode }
% arara: clean: { extensions: [ aux, blg, idx, ilg, ind, log, out, pytxcode, rel, toc ] }
% !arara: clean: { files: [ ans.tex, hint.tex] }

% arara: pdflatex
% arara: clean: { extensions: [ aux, blg, idx, ilg, ind, log, out, pytxcode, rel, toc ] }
% !arara: clean: { files: [ ans.tex, hint.tex] }


\documentclass[queueing-book-solution.tex]{subfiles}
%\externaldocument{queueing-book}

\opt{solutionfiles,check}{
\loadgeometry{tufte}
\Opensolutionfile{hint}
\Opensolutionfile{ans}
}

\begin{document}

\section{$M/M/1$ queue and its variations}
\label{sec:mm1}


In this section we devevelop many different queueing models by using the level-crossing equations~\cref{eq:25} and a judicious choice of $\lambda(n)$ and $\mu(n)$. However,  we need to require that the time to the next arrival or departure is memoryless\sidenote{hence, exponential}. Here we present the results; the exercises ask you to derive the formulas.\sidenote{The main challenge is not to make computational errors} In~\cref{sec:mnmn1} we show how to apply the models we derive here.

As said,  the inter-arrival times and service times need to be memoryless.
Specifically,  this means that\sidenote{Compare the  construction of the Poisson process in~\cref{ex:p-355}.}
\begin{align*}
  \P{\L(t+\Delta t) = n+1 \given \L(t) = n } &= \lambda(n) \Delta t + o(\Delta t), \quad n\geq 0,\\
  \P{\L(t+\Delta t) = n-1 \given \L(t) = n } &= \mu(n) \Delta t + o(\Delta t), \quad n \geq 1.
\end{align*}
In other words, the arrival rate $\lambda(n)$ and service rate $\mu(n)$ may depend on the number of jobs in the system, but not more.


\newthought{For the \recall{$M/M/1$} queue} we take~$\lambda(n)=\lambda$ for the arrival rate and $\mu(n)=\mu$ for the service rate.
Therefore, the level-crossing equations become
\begin{equation*}
 p(n+1) = \frac{\lambda(n)}{\mu(n+1)} p(n) = \frac{\lambda}{\mu} p(n) = \rho p(n),\quad \rho = \frac{\lambda}{\mu}.
\end{equation*}
As this holds for any $n\geq 0$, $p(n+1) = \rho^{n+1} p(0)$.
Normalization with~\cref{eq:20} and~\cref{eq:61} gives that $p(0) =1-\rho$, hence
\begin{align}\label{eq:23}
p(n) &= (1-\rho)\rho^{n}, \quad n \geq 0.
\end{align}

It is now easy to compute the most important performance measures.
The utilization is $\rho=\lambda/\mu$, and with a bit of algebra,\sidenote{\cref{ex:34}}
\begin{align}\label{eq:el}
\E\Ls &= \rho, & \E \L &= \frac \rho{1-\rho}, & \P{\L> n} &= \rho^{n+1},
\end{align}
recall that $\E\Ls$ is the time-average number of jobs in service, $\E\L$ is the number in the system. For the number in queue,
\begin{equation*}
\E \Q = \E \L - \E Ls = \frac{\rho^2}{1-\rho}.
\end{equation*}


\newthought{In the \recall{$M/M/1/K$} queue}  jobs are blocked\sidenote{Or they are not willing to join the system} when $L\geq K$.
We can implement this by taking $\lambda(n) = \lambda \1{n < K}$, which
gives $\lambda p(n) = \mu p(n+1)$ for $n<K$, and $p(n) = 0$ for $n>K$.
Then, using~\cref{eq:61}, it follows right away  that
 \begin{equation*}
p(n) = \frac{1-\rho}{1-\rho^{K+1}} \rho^n, \quad \rho = \frac{\lambda}{\mu}.
\end{equation*}
Obseve that, since jobs are blocked, the utilization is no longer equal to the load. The utilization is now $\lambda p(K)/\mu \neq \lambda /\mu = \rho$. Note further that, contrary to the $M/M/1$ case,
\begin{equation*}
\E\Ls = \sum_{n=0}^K \min\{n, 1\} p(n) = 1-p(0) \neq \rho.
\end{equation*}
The rest of the performance measures follow also easily.

\newthought{Consider next the \recall{$M/M/c$} queue}.
For this, we set $\lambda(n) = \lambda$, and $\mu(n) = \mu \min\{n, c\}$.
As in~\cref{eq:6}, we take $\rho=\lambda/(c\mu)$.
After some calculations we obtain\sidenote{\cref{ex:7}}
\begin{subequations}
\begin{align}
p(n) &= \frac 1 {G} \frac{1}{\Pi_{k=1}^{n}\min\{c, k\}}(c\rho)^{n} , \quad n\geq 1,\label{eq:1}\\
G &= \frac 1{p(0)} = \sum_{n=0}^{c-1} \frac{(c\rho)^n}{n!} + \frac{(c\rho)^c}{(1-\rho)c!}, \label{eq:501}\\
\E{\Q} &= \sum_{n=c}^\infty (n-c) p(n) = \frac{(c\rho)^c}{c! G}\frac{\rho}{(1-\rho)^2}\\
\E{\Ls} &= \sum_{n=0}^{\infty}\min\{n,c\} p(n)= \frac{\lambda}\mu.
\end{align}
\end{subequations}
Observe that the expected number of busy servers $\E{\Ls}$ is equal to the load $\lambda \E S=\lambda/\mu$ (and not to $\rho$).


\newthought{The  \recall{$M/M/c/K$} queue} is a multi-server queue with blocking and is a simple generalization of the $M/M/c$ and $M/M/1/K$ queue.
Now, $\lambda(n) = \lambda\1{n<K}$ and $\mu(n) = \mu \min\{n, c\}$.
The probabilities $p(n)$ are given by~\cref{eq:1}. Mind again that now the utilization is not the same as the load.
The computation of the normalization is numerically trivial.


Here is an example in code.\sidenote{The other models of this section are just as easily implemented.}  We use just the recursion $\lambda(n) p(n) = \mu(n+1)p(n+1)$ to compute $p(n)$, as this is easier to code, more generic, and, in this case, also numerically more efficient.

\begin{pyconsole}
import numpy as np

mu, c, K = 2, 4, 8
labda = 3 * np.ones(K)
mu = np.arange(K + 1)
mu = np.minimum(mu, c)

p = np.ones(K + 1, dtype=float)

for n in range(K):
    p[n + 1] = p[n] * labda[n] / mu[n + 1]

p /= p.sum()  # normalize
ELs = sum(p[n] * min(n, c) for n in range(len(p)))
\end{pyconsole}

\newthought{The \recall{$M/M/c/c$} queue} is a special case of the $M/M/c/K$ queue: the number of servers is exactly equal to the number of jobs that are allowed to enter the system.\sidenote{This queueing model is also known as the Erlang $B$ model.} This system is often used to determine the number of beds needed by a hospital: the beds act as servers and the patients as jobs.
Given the arrival rate of patients, and the average time they occupy a bed\sidenote{i.e., the expected service time}, the problem is to find the number of beds $c$ such that the loss probability $p(c)$ is less than some threshold, $1\%$ say.

For hospitals it is reasonable to assume  Poisson arrivals since patients arrive independently from a large population.\sidenote{PASTA implies $\pi(c) = p(c)$, hence the loss fraction is indeed $p(c)$.}
Also, there are typically many patients in the hospital, hence the recovery times are quite accurately described by an exponential distribution.

Take $\lambda(n) = \lambda\1{n<c}$, and $\mu(n) = n \mu$.
As for the $M/M/c$ queue, $\rho = \lambda/(c \mu)$.
Then some algebra\sidenote{\cref{ex:l-245}} gives
 \begin{align*}
   p(n) &= \frac 1 G \frac{(c\rho)^n}{n!}, 0\leq n \leq c, &
   G&=\sum_{n=0}^{c} \frac{(c\rho)^n}{n!},\\
   \E \Q &= 0, &
\E \Ls &= \frac{\lambda}{\mu} \left(1- p(c)\right).
\end{align*}
The last two results are easy.
As there are as many servers as jobs, jobs cannot be in queue.
Next, observing that $\lambda(1-p(c))$ is the rate of accepted jobs\sidenote{Since, by PASTA, a fraction $p(c)$ is lost.}, the load is $\lambda(1-p(c))/\mu$, and this in turn is equal to the average number of busy servers.


\newthought{The \recall{$M/M/\infty$} queue} is simple to analyze, as is has \emph{ample} servers.
Any job that arrives finds a free server, hence, its service can start right away.
Therefore, $\lambda(n)=\lambda$ and $\mu(n) = n \mu$ for all $n\geq 0$.
By taking the limit $c\to \infty$ in the expressions of the $M/M/c$ queue (or the $M/M/c/c$ queue) we get
\begin{align*}
  p(n) &= e^{-\lambda/\mu} \frac{(\lambda/\mu)^n}{n!}, & \E \L = \E \Ls = \frac \lambda \mu.
\end{align*}
We see that the number of busy servers in the $M/M/\infty$ queue is Poisson distributed with parameter $\lambda \E S$.
We mention in passing---but do not prove it---that the same results also hold for the $M/G/\infty$ queue.

\newthought{With a finite calling population} the number of jobs is fixed, $N$ say.
This model is useful to analyze repair systems and inventory systems of spare parts.
To see this, consider a factory with $N$ machines and one mechanic.\sidenote{When there are also $N$ servers available, we obtain the Ehrenfest model of diffusion, which is used to explain the second law of thermodynamics.}
A machine can be in one of two states: working or failed.
When a machine breaks down, it moves to the repair department and waits until it is repaired.
When $n$ machines are in repair, there are $N-n$ machines still working.
Thus, if $\lambda$ is the rate at which a machine can fail, $\lambda(N-n)$ is the rate at which any of the working machines can fail.
Since there is one mechanic: $\mu(n)=\mu$.


With this model for $\lambda(n)$ and $\mu(n)$ the probabilities become \sidenote{~\cref{ex:l-247}}
\begin{align*}
 p(n)  &= \frac{N!}{(N-n)!}\frac{\rho^n}{G}, & G &= \sum_{n=0}^N \rho^n \frac{N!}{(N-n)!}.
\end{align*}
As the expression for $G$ cannot be simplified, there is not much point trying to derive simple expressions for $\E \L$.


\newthought{Balking customers leave} when they find the queue too long at the moment they arrive.
A simple example model with customer balking is $\mu(n)=\mu$ and
 \begin{equation*}
 \lambda(n) =
 \begin{cases}
 \lambda, &\text{ if } n=0, \\
 \lambda/2, &\text{ if } n=1, \\
 0, &\text{ if } n > 1.
\end{cases}
\end{equation*}

Balking is not necessarily the same as blocking.
In the latter case, $\lambda(n)= \lambda \1{n < K}$; in the former case, a fraction of the customers may already choose not the join the system at a lower level than $K$.

As the steady-state probabilities depend entirely on the form of $\lambda(n)$ and $\mu(n)$, we have to use the recursion~\cref{eq:25} and the code above to compute $p(n)$.
With this, the  rate at which customers \emph{enter} the system becomes $\lambda = \sum_{n=0}^\infty \lambda(n) p(n)$, and this is the arrival rate we need to use in Little's law.\sidenote{\cref{ex:39}}


\begin{exercise}\label{ex:34}
Use moment-generating functions to derive $\E\L$ and $\V\L$ for the $M/M/1$ queue, and show that $\P{L>n} = \rho^{n+1}$.
\begin{hint}
First show that $M_L(s) = (1-\rho) \sum_n e^{s n} \rho^n$, then use~\cref{eq:61}.
Similarly, $\P{\L\geq n} = \sum_{k\geq n} p(k)$.
\end{hint}
\begin{solution}
\begin{equation*}
 M_L(s) = \E{e^{s L}} = \sum_{n=0}^\infty e^{s n}p(n) = (1-\rho) \sum_n e^{s n} \rho^n=\frac{1-\rho}{1-e^{s}\rho},
\end{equation*}
where we assume that $s$ is such that $e^s \rho < 1$. Then,
\begin{align*}
 M_L'(s) &= (1-\rho) \frac{1}{(1-e^s\rho)^2} e^s \rho \implies \E\L = M_L'(0)= \frac{\rho}{1-\rho}, \\
 \E{\L^2}&= M''(0)= \frac{2\rho^2}{(1-\rho)^2} + \frac{\rho}{1-\rho},\\
\V\L &= \E{L^2} - (\E\L)^2 = \frac{\rho(1+\rho)}{(1-\rho)^2}-\frac{\rho^2}{(1-\rho)^2} = \frac{\rho}{(1-\rho)^2},\\
 \P{\L\geq n}
 &= \sum_{k=n}^\infty p(k) = \sum_{k=n}^\infty p(0)\rho^k = (1-\rho)\sum_{k=n}^\infty \rho^k
= (1-\rho)\rho^n \sum_{k=0}^\infty\rho^k = (1-\rho) \rho^n \frac1{1-\rho} = \rho^n.
\end{align*}
\end{solution}
\end{exercise}

\begin{exercise}
Numerically check that in \cref{ex:l-185}, $\beta=1-\E\Ls \mu /\lambda$ for the $M/M/3/8$ queue with $\lambda=3$ and $\mu=2$.
\begin{solution}
Use the results of \cref{ex:41}; use PASTA to conclude that $\beta=p(K)$.
\end{solution}
\end{exercise}




\begin{exercise}
 Explain that for the $M/M/1$ queue $\E{\Q} = \sum_{n=1}^\infty (n-1)\pi(n)$ and use this to find that $\E{\Q}=\rho^2/(1-\rho)$.
\begin{solution}
 The fraction of time the system contains $n$ jobs is $\pi(n)$ (by
 PASTA). When the system contains $n>0$ jobs, the number in queue
 is one less, i.e., $n-1$.
\begin{align*}
\E{\Q}
&= \sum_{n=1}^\infty (n-1)\pi(n)
=\sum_{n=1}^\infty n\pi(n) -\sum_{n=1}^\infty \pi(n)
= \E\L - (1-\pi(0)) = \E\L - \rho.
\end{align*}
\end{solution}
\end{exercise}



\begin{exercise}\label{ex:7}
Derive the expressions for the  $M/M/c$ queue.
\begin{solution}
  For $n\geq 1$,
 \begin{align*}
 p(n)
 &= \frac{\lambda(n-1)}{\mu(n)}p(n-1)
 = \frac{\lambda}{\min\{c, n\} \mu }p(n-1)
 = \frac{1}{\min\{c, n\}}(c\rho) p(n-1) \\
 & = \frac{1}{\min\{c, n\}\min\{c, n-1\}}(c\rho)^2 p(n-2) \\
 &= \frac{1}{\Pi_{k=1}^{n}\min\{c, k\}}(c\rho)^{n} p(0).
 \end{align*}


To obtain the normalization constant $G = 1/p(0)$,
\begin{align*}
1 &= \sum_{n=0}^\infty p(n)
= p(0) + \sum_{n=1}^{c-1} p(n) + \sum_{n=c}^\infty p(n) \\
&=p(0) + p(0) \sum_{n=1}^{c-1}\frac{(c\rho)^n}{n!} +
 p(0)\sum_{n=c}^{\infty} \frac{c^c}{c!} \rho^{n} \\
&=p(0)\sum_{n=0}^{c-1}\frac{(c\rho)^n}{n!} +
 p(0) \sum_{n=c}^{\infty} \frac{(c\rho)^c}{c!} \rho^{n-c} \\
&=
p(0)\sum_{n=0}^{c-1}\frac{(c\rho)^n}{n!} +
p(0)\frac{(c\rho)^c}{c!} \sum_{n=0}^{\infty} \rho^n \\
&=
p(0) \sum_{n=0}^{c-1}\frac{(c\rho)^n}{n!} +
p(0)\frac{(c\rho)^c}{c!(1-\rho)}.
\end{align*}
Next,
\begin{align*}
 \E{\Q}
&=\sum_{n=c}^\infty (n-c) p(n)
=\sum_{n=c}^\infty (n-c) \frac{c^c}{c!}\rho^{n} p(0) \\
&=\frac{c^c\rho^c}{G c!} \sum_{n=c}^\infty (n-c) \rho^{n-c}
=\frac{c^c\rho^c}{G c!} \sum_{n=0}^\infty n \rho^n \\
&=\frac{c^c\rho^c}{G c!} \frac{\rho}{(1-\rho)^2}.
\end{align*}

The derivation of the expected number of jobs in service becomes easier if we pre-multiply the normalization constant $G$:
 \begin{align*}
 G \E{\Ls}
&= G \left( \sum_{n=0}^{c} n p(n) + \sum_{n=c+1}^{\infty} c p(n) \right) \\
&= \sum_{n=1}^{c} n \frac{(c\rho)^n}{n!} + \sum_{n=c+1}^{\infty} c \frac{c^c\rho^n}{c!}
= \sum_{n=1}^{c} \frac{(c\rho)^n}{(n-1)!} + \frac{c^{c+1}}{c!}\sum_{n=c+1}^{\infty} \rho^n\\
&= \sum_{n=0}^{c-1} \frac{(c\rho)^{n+1}}{n!} + \frac{(c\rho)^{c+1}}{c!}\sum_{n=0}^{\infty} \rho^n
= c\rho \left(\sum_{n=0}^{c-1} \frac{(c\rho)^n}{n!} + \frac{(c\rho)^{c}}{c!(1-\rho)}\right).
 \end{align*}
Observe that the RHS is precisely equal to $\rho c G$, so that $\E\Ls = c \rho$.
\end{solution}
\end{exercise}


\begin{exercise}
 Check that the performance measures of the $M/M/c$ queue reduce to those of the $M/M/1$ queue if $c=1$.
\begin{hint}
Fill in $c=1$. Realize that this is a check on the formulas.
\end{hint}
\begin{solution}
Take $c=1$
%\begin{subequations}
 \begin{align*}
p(0) &= \frac{1}G \frac{(c\rho)^0}{0!}=\frac1 G, \\
p(n) &= \frac{1}G \frac{c^c\rho^n}{c!} = \frac{1}G \frac{1^1\rho^n}{1!} =\frac{\rho^n}G , \quad n=1,2, \ldots \\
G &=\sum_{n=0}^{c-1} \frac{(c\rho)^n}{n!} + \frac{(c\rho)^c}{(1-\rho)c!}
=\sum_{n=0}^{0} \frac{\rho^0}{0!} + \frac{\rho}{(1-\rho)} = 1 + \frac{\rho}{1-\rho} = \frac1{1-\rho},
\\
\E{\L} &= \frac{(c\rho)^c}{c! G}\frac{\rho}{(1-\rho)^2} = \frac{\rho}{1/(1-\rho)}\frac{\rho}{(1-\rho)^2} = \frac{\rho^2}{1-\rho}, \\
\E{\L} &= \sum_{n=0}^{c}n p(n) + \sum_{n=c+1}^\infty c p(n) = p(1) + 1 \sum_{n=2}^\infty p(n) = 1- p(0) = \rho.
\end{align*}
%\end{subequations}
Everything is in accordance to the formulas we derived earlier for the $M/M/1$ queue.
\end{solution}
\end{exercise}



\begin{exercise}\label{ex:l-245}
Derive the expressions for the $M/M/c/c$ queue.
\begin{hint}
  Realize that the $M/M/c/c$ queue is similar to the $M/M/c$ queue. However, there cannot be more than $c$ jobs in the system.
\end{hint}
\begin{solution}
In the expressions for the $M/M/c$ queue, just neglect the parts that deal with states $n>c$.
We see that $p(n) = p(0)(c\rho)^n/{n!}$. For the normalization
$1=\sum_{n=0}^c p(n) = p(0) \sum_{n=0}^c \frac{(c\rho)^n}{n!}$. Next,
 \begin{align*}
 \E{\Ls}
&= \sum_{n=0}^{c} n p(n) = \sum_{n=1}^c n p(n) \\
&= G^{-1} \sum_{n=1}^c n \frac{(\lambda/\mu)^n}{n!}
= G^{-1} \sum_{n=1}^{c} \frac{(\lambda/\mu)^{n}}{(n-1)!} \\
&= \frac{\lambda}{\mu G} \sum_{n=0}^{c-1} \frac{(\lambda/\mu)^{n}}{n!}
= \frac{\lambda}{G\mu} \left(G- \frac{(\lambda/\mu)^c}{c!}\right) \\
&= \frac{\lambda}{\mu} \left(1- \frac{1}G\frac{(\lambda/\mu)^c}{c!}\right) \\
&= \frac{\lambda}{\mu} \left(1- p(c)\right).
 \end{align*}
\end{solution}
\end{exercise}

\begin{exercise}\label{ex:l-246}
Derive the expressions for the $M/M/\infty$ queue.
\begin{hint}
Use that for any $x$, $x^n/n!\to 0$ as $n\to\infty$.
\end{hint}
\begin{solution}
 By taking the limit $c\to\infty$, note first that in~\cref{eq:501},
\begin{equation*}
\frac{(c\rho)^c}{(1-\rho)c!} = \frac{(\lambda/\mu)^c}{(1-\rho)c!}\to 0, \quad\text{as } c\to \infty.
\end{equation*}
Hence
\begin{equation*}
G =\sum_{n=0}^{c-1} \frac{(c\rho)^n}{n!} + \frac{(c\rho)^c}{(1-\rho)c!} \to \sum_{n=0}^{\infty} \frac{(c\rho)^n}{n!} = e^{\lambda/\mu}.
\end{equation*}
Next, for any fixed $n$, eventually $c>n$, and then, as $\rho=\lambda/(\mu c)$,
\begin{equation*}
 p(n) = \frac{1}G \frac{(c\rho)^n}{n!} = \frac{1}G \frac{(\lambda/\mu)^n}{n!}
\to e^{-\lambda/\mu} \frac{(\lambda/\mu)^n}{n!}, \quad\text{as } c\to\infty.
\end{equation*}
\end{solution}
\end{exercise}





\begin{exercise}\label{ex:l-247}
Find  the steady state probabilities  for a single-server queue with a finite calling population with $N$ jobs.
\begin{solution}
 \begin{align*}
 p(n+1)
& = \frac{(N-n)\lambda}\mu p(n)
 = \rho (N-n) p(n) \\
& = \rho^2 (N-n)(N-(n-1))p(n-1) \\
& = \rho^3 (N-n)(N-(n-1))(N-(n-2)) p(n-2) \\
& = \rho^{n+1} (N-n)(N-(n-1))\cdots(N-(0)) p(0) \\
&= \rho^{n+1} \frac{N!}{(N-(n+1))!}p(0).
 \end{align*}
\end{solution}
\end{exercise}

\begin{exercise}\label{ex:33}
 Derive the steady state probabilities $p(n)$ for a queue with a finite calling population with $N$ jobs and $N$ servers.
  What happens if $N\to\infty$?
\begin{solution}
 Take $\lambda(n) = (N-n)\lambda$ and $\mu(n) = n \mu$. Then
 \begin{align*}
 p(n+1)
&= \frac{\lambda(n)}{\mu(n+1)} p(n)
= \frac{(N-n)\lambda}{(n+1)\mu} p(n)
= \frac{(N-n)(N-(n-1))}{(n+1)n}\frac{\lambda^2}{\mu^2} p(n-1) \\
&= \frac{N!}{(N-(n+1))!}\frac1{(n+1)!}\rho^{n+1} p(0)
 = {N \choose n+1}\rho^{n+1} p(0).
 \end{align*}
Therefore, $G=\sum_{k=0}^N \rho^k { N \choose k}$.
\end{solution}
\end{exercise}



\begin{exercise}\label{ex:40}
 Show that as $K\to\infty$, the performance measures of the $M/M/1/K$ converge to those of the $M/M/1$ queue.
\begin{hint}
Use that $\sum_{i=0}^n x^i = (1-x^{n+1})/(1-x)$. BTW, is it
 necessary for this expression to be true that $|x|<1$? What should
 you require for $|x|$ when you want to take the limit
 $n\to\infty$?
\end{hint}
\begin{solution}
To take the limit $K\to\infty$---mind, not the limit $n\to\infty$---, write
\begin{equation*}
G= \frac{1-\rho^{K+1}}{1-\rho} = \frac{1}{1-\rho} -\frac{\rho^{K+1}}{1-\rho}.
\end{equation*}
Since $\rho^{K+1}\to 0$ as $K\to \infty$ (recall, $\rho<1$), we get
\begin{equation*}
G \to \frac{1}{1-\rho},
\end{equation*}
as $K\to\infty$. Therefore, $p(n)=\rho^n/G \to \rho^n(1-\rho)$, and
the latter are the steady-state probabilities of the $M/M/1$
queue. Finally, if the steady-state probabilities are the same, the
performance measures (which are derived from $p(n)$) must be the same.
\end{solution}
\end{exercise}


\begin{exercise}
Derive the expression for $\E\L$ in~\cref{eq:el} by means of indicator variables.
\begin{solution}
\begin{align*}
\E\L &= \sum_{n=0}^\infty n p(n) = \sum_{n=0}^\infty \sum_{i=1}^n \1{i\leq n} p(n) && n=\sum_{i=1}^n \1{i\leq n}\\
&= \sum_{n=0}^\infty \sum_{i=1}^\infty \1{i\leq n} p(n) && i>n\implies \1{i\leq n} = 0\\
&= \sum_{i=1}^\infty \sum_{n=0}^\infty \1{i\leq n} p(n) = \sum_{i=1}^\infty \sum_{n=i}^\infty p(n) && n < i \implies \1{i\leq n}=0 \\
&= \sum_{i=1}^\infty \sum_{n=i}^\infty (1-\rho)\rho^n && p(n) = (1-\rho)\rho^n \\
&= \sum_{i=1}^\infty \sum_{n=0}^\infty (1-\rho)\rho^{n+i} && n\to n+i \\
&= \sum_{i=1}^\infty (1-\rho)\rho^i \sum_{n=0}^\infty \rho^n && \rho^{n+i}=\rho^i \rho^n\\
&= \sum_{i=1}^\infty (1-\rho)\rho^i \frac1{1-\rho} = \sum_{i=1}^\infty \rho^i = \sum_{i=0}^\infty \rho^{i+1} && i\to i+1\\
&= \rho \sum_{i=0}^\infty \rho^i = \frac{\rho}{1-\rho}.
\end{align*}
Note that, since the summands are positive, we can use Fubini's theorem
to justify the interchange of the summations.
\end{solution}
\end{exercise}


\begin{exercise}
Derive $\E\L$ and $\E{\L^2}$  by differentiating the  LHS and RHS of $\sum_{n=0}^{\infty}\rho^n = (1-\rho)^{-1}$.
\begin{solution}
  Differentiate the left- and RHS of $(1-\rho)^{-1} = \sum_{n=0}^\infty \rho^n$ with respect to $\rho$ and then multiply with $\rho$ to get
\begin{equation*}
\dfrac{\rho}{(1-\rho)^2}=\sum_{n=0}^{\infty}n\rho^n.
\end{equation*}
Then multiply both sides by $1-\rho$ and use that $p(n) = (1-\rho)\rho^n$ to get $\E\L$.

Differentiating and multiplying with $\rho$ a second time yields
\begin{align*}
\rho \frac{(1-\rho)^2 + \rho2(1-\rho)}{(1-\rho)^4} &= \rho \frac{1-2\rho+\rho^2 + 2\rho-2\rho^2}{(1-\rho)^4}
                                                     = \rho \frac{1-\rho^2}{(1-\rho)^4} \\
  &=\rho \dfrac{1+\rho}{(1-\rho)^3}=\sum_{n=0}^{\infty}n^2\rho^n,
\end{align*}
and hence for $\E{\L^2}$,
\begin{align*}
  (1-\rho)\sum_{n=0}^{\infty}n^2\rho^n
  &= \rho\dfrac{1+\rho}{(1-\rho)^2}  = \dfrac{\rho}{(1-\rho)^2} + \dfrac{\rho^2}{(1-\rho)^2}
    = \dfrac{2\rho^2}{(1-\rho)^2} + \dfrac{\rho}{(1-\rho)^2} - \dfrac{\rho^2}{(1-\rho)^2} \\
  &= \dfrac{2\rho^2}{(1-\rho)^2} + \rho\dfrac{(1-\rho)}{(1-\rho)^2} = \dfrac{2\rho^2}{(1-\rho)^2} + \dfrac{\rho}{1-\rho}.
\end{align*}
Recall that $p(n) = (1-\rho)\rho^n$.
\end{solution}
\end{exercise}

\opt{solutionfiles}{\Closesolutionfile{hint}
\Closesolutionfile{ans}
\loadgeometry{normal}
\input{hint}
\input{ans}
}

\end{document}



%%% Local Variables:
%%% mode: latex
%%% TeX-master: t
%%% End:
