% arara: pdflatex: { shell: yes, interaction: nonstopmode }
% arara: pythontex: {verbose: yes, rerun: modified }
% arara: pdflatex: { shell: yes, interaction: nonstopmode }
% arara: clean: { extensions: [ aux, blg, idx, ilg, ind, log, out, pytxcode, rel, toc ] }
% !arara: clean: { files: [ ans.tex, hint.tex] }

% arara: pdflatex
% arara: clean: { extensions: [ aux, blg, idx, ilg, ind, log, out, pytxcode, rel, toc ] }
% !arara: clean: { files: [ ans.tex, hint.tex] }


\documentclass[queueing-book-solution.tex]{subfiles}
%\externaldocument{queueing-book}

\opt{solutionfiles,check}{
\loadgeometry{tufte}
\Opensolutionfile{hint}
\Opensolutionfile{ans}
}

\begin{document}


\section{Rate, Stability and Load}
\label{sec:rate-stability}



In this section, we develop a number of measures to characterize the performance of queueing systems in steady-state.
In particular, we define the load, which is, arguably, the most important performance measure of a queueing system to check.




\newthought{We first formalize} the arrival rate and departure rate in terms of the arrival and departure processes $\{A(t)\}$ and $\{D(t)\}$, see~\cref{sec:constr-gg1-queu}.
The \recall{arrival rate} is the long-run average number of jobs that arrive per unit time along a sample path, i.e.,\sidenote{This limit does not necessarily exist if $A(t)$ is some pathological function.
}
\begin{equation}\label{eq:3}
 \lambda = \lim_{t\to\infty} \frac{A(t)}t.
\end{equation}
Likewise, the \recall{departure rate} is
\begin{equation}\label{eq:28}
 \delta = \lim_{t\to\infty} \frac{D(t)}t.
\end{equation}
Henceforth we assume that both limits are finite.

\newthought{Observe that,} if the system is empty\sidenote{Why is this necessary to require?} at time $0$, i.e., $L(0)=0$,
 the number of departures must be smaller than or equal to the number of arrivals, i.e., $D(t) \leq A(t)$ for all $t$.
Therefore,
\begin{equation*}
\delta = \lim_{t\to\infty} \frac{D(t)}t \leq \lim_{t\to\infty} \frac{A(t)}t = \lambda.
\end{equation*}
It is intuitively obvious that when $\lambda > \delta$, the system length process $L(t)=L(0)+A(t) - D(t) \to \infty$ as $t\to \infty$.
Combining the above two inequalties, we say that a system is \recall{rate-stable} if
\begin{equation}\label{eq:84}
 \lambda = \delta.
\end{equation}
In  words: the system is rate-stable whenever jobs leave the system just as fast as they arrive in the long run.

\newthought{Suppose that we} are given a sequence of iid inter-arrival times  $\{X_k\}$  distributed as the common rv~$X$. Then we can relate the arrival rate $\lambda$ to the mean inter-arrival time $\E X$ as follows.\sidenote{The existence of the limit~\cref{eq:3} is  also guaranteed under this assumption.}
Observe that at time $t=A_n$ precisely $n$ arrivals occurred.
Then, with~\cref{ex:61}
% by applying the definition of $A(t)$ at the epochs $A_n$,
we see that $A(A_n) = n$, and therefore,
\begin{equation*}
 \frac{1}n\sum_{k=1}^n X_k = \frac{A_n}n = \frac{A_n}{A(A_n)}.
\end{equation*}
Below we show that $A_n\to\infty$ as $n\to\infty$ implies that $\lim_{n\to\infty} A_n/A(A_{n)} = \lim_{t\to\infty} t/A(t)$. With this, it follows from~\cref{eq:3} that the average inter-arrival time between two consecutive jobs is
\begin{equation}\label{eq:54}
 \E X = \lim_{n\to\infty} \frac{1}n\sum_{k=1}^n X_k = \lim_{n\to \infty} \frac{A_n}{A(A_n)} = \lim_{t\to\infty} \frac t{A(t)} = \frac 1 \lambda,
\end{equation}
where we take $t=A_n$ in the limit for $t\to\infty$.
In words, the arrival rate $\lambda$ is the \emph{inverse} of the expected inter-arrival time $\E X$.


\newthought{In~\cref{eq:54} we replace} the limit with respect to $n$ by a limit with respect to~$t$.
To show that this is allowed, observe that $A_{A(t)}$ is the arrival time of the last job before time $t$ and that $A_{A(t)+1}$ is the arrival time of the first job after time $t$.
Therefore,
 \begin{equation*}
 A_{A(t)} \leq t < A_{A(t)+1} \Leftrightarrow
 \frac{A_{A(t)}} {A(t)} \leq \frac{t}{A(t)} <\frac{A_{A(t)+1}}{A(t)} = \frac{A_{A(t)+1}}{A(t)+1}\frac{A(t)+1}{A(t)}.
 \end{equation*}
 Now $A(t)$ is a counting process such that $A(t)\to\infty$ as $t\to\infty$, therefore,
 \begin{equation*}
\lim_{n\to\infty} \frac{A_n}{n} = \lim_{t\to\infty} \frac{A_{A(t)}}{A(t)} = \lim_{t\to\infty} \frac{A_{A(t)+1}}{A(t)+1},
 \end{equation*}
 where the third limit follows trivially from the second.
 Finally, because $(A(t)+1)/A(t)\to 1$, we arrive at the equality $\lim_{t\to\infty} t/A(t) = \lim_{n\to\infty} A_n/n$.

\newthought{Consider the $G/G/1$ queue}.\sidenote{See~\cref{ex:l-164} for an extension to $G/G/c$ queues.}
Let $S_k$ be the required service time of the $k$th job to be served, so that $U_n = \sum_{k=1}^n S_k$ is the total service time of the first $n$ jobs.
Letting $ U(t) = \max\{n: U_n \leq t\}$, we define the \recall{service}, or \recall{processing}, rate as
\begin{equation*}
 \mu = \lim_{t\to\infty} \frac{U(t)}t.
\end{equation*}
Similar to the relation  $\E X= 1/\lambda$, we have the relation
\begin{equation*}
 \E S = \lim_{n\to\infty} \frac 1 n \sum_{k=1}^n S_k = \lim_{n\to\infty} \frac{U_n}{n} = \lim_{n\to\infty} \frac{U_n}{U(U_n)} = \lim_{t\to\infty} \frac t{U(t)} = \frac 1 \mu.
\end{equation*}
% Thus, the service rate $\mu$ is the \emph{inverse} of $\E S$.




\newthought{Once we have } the arrival and service rate, we define the \recall{load} as  the rate at which work arrives:
\begin{equation*}
\textrm{Load } =  \lambda \E S =\frac{\lambda}{\mu} = \frac{\E S}{\E X}.
\end{equation*}
For the $G/G/c$ queue, we define the \recall{utilization} as,\sidenote{Pay attention, only for the $G/G/1$ queue, the load is equal to the utilization.}
\begin{equation*}
\textrm{Utilization } = \rho=\lambda \E S /c.
\end{equation*}

It is easy to check  with a simulation of the $G/G/1$ queue that $L(t)$ increases at rate $\lambda-\mu$ when $\lambda > \mu$, while $L(t) \approx L(0) + (\lambda - \mu)t$ when $\lambda< \mu$ and $L(0)$ large, until the system is empty.\sidenote{It turns out that, when $\E X = \E S$ but $\V{X-S} > 0$, the queue length process behaves in a very peculiar way.
  This is due to the fact that the symmetric random walk has some unexpected behavior, and~\cref{sec:queu-proc-as} shows that queueing systems and random walks are intimately related.}

For this reason the utilization is, perhaps, the most important performance measure to check: when $\rho\geq 1$, we are `in trouble', when $\rho < 1$, we are `safe'.
In the sequel we tacitly assumte that $\rho<1$, unless stated otherwise.



\begin{exercise}
  Can you make an arrival process such that $A(t)/t$ does not have a limit?
\begin{hint}
  As a start, the function $\sin(t)$ does not have a limit as $t\to\infty$.
  However, the time-average $\sin(t)/t \to 0$.
  Now you need to make some function whose time-average does not converge, hence it should grow fast, or fluctuate wilder and wilder.
\end{hint}
\begin{solution}
 If $A(t) = 3 t^2$, then clearly $A(t)/t = 3t$. This does not
 converge to a limit.

 Another example, let the arrival rate $\lambda(t)$ be given as follows:
 \begin{equation*}
 \lambda(t) =
 \begin{cases}
 1 & \text{if } 2^{2k} \leq t < 2^{2k+1} \\
 0 & \text{if } 2^{2k+1} \leq t < 2^{2(k+1)},
 \end{cases}
 \end{equation*}
 for $k=0,1,2,\ldots$.
 Let $A(t) = \lambda(t) t$.
 Then $A(t)/t$ does not have a limit.
 Of course, these examples are quite pathological, and are not representable for `real life cases'.
 (Although this is also quite vague.
 What, then, is a real-life case?)

 For the mathematically interested, we seek a function whose Ces\`aro limit does not exist.
\end{solution}
\end{exercise}


\begin{exercise}\label{ex:98}
If the system starts empty, then we know that the number $L(t)$ in the system at time $t$ is equal to $A(t) - D(t)$.
Show that the system is rate-stable  if $L(t)$ remains finite, or, more generally, $L(t)/t \to 0$ as $t\to\infty$.
\begin{solution}
Since $L(t) = L(0) + A(t) - D(t)$,
\begin{equation*}
 \lambda = \lim_{t \to \infty} \frac{A(t)}t = \lim_{t \to \infty} \frac{D(t)+L(t)}t = \lim_{t \to \infty} \frac{D(t)}t + \lim_{t \to \infty} \frac{\L(t)}t
 = \delta.
\end{equation*}
Hence, $\lambda=\delta$ when $L(t)/t\to0$.
\end{solution}
\end{exercise}




\begin{exercise}\label{ex:l-253}
 Show
that $\E{ X_k-S_k} > 0$ implies that $\rho < 1$.
\begin{hint}
Remember that $\{X_k\}$ and $\{S_k\}$ are sequences of iid random variables. What are the implications for the expectations?
\end{hint}
\begin{solution}
 $0>  \E {S_{k}-X_k} = \E{ S_{k}}- \E {X_k} = \E S - \E X$, where we use the fact that the $\{S_k\}$ and $\{X_k\}$ are iid sequences. Hence,
 \begin{equation*}
 \E X > \E S \iff \frac 1{\E S} > \frac1{\E X} \iff \mu > \lambda.
 \end{equation*}

\end{solution}
\end{exercise}


\begin{exercise}\label{ex:l-164}
 Consider a queueing system with $c$ servers with identical production rates $\mu$.
 What would be a reasonable stability criterion for this system?
\begin{hint}
What is the rate in, and what is the service capacity?
\end{hint}
\begin{solution}
 The criterion is that $c$ must be such that $\lambda < c\mu$.
 (Thus, we interpret the number of servers as a \emph{control}, i.e., a `thing' we can change, while we assume that $\lambda$ and $\mu$ cannot be easily changed.)
 To see this, we can take two different points of view.
 Imagine that the $c$ servers are replaced by one server that works $c$ times as fast.
 The service capacity of these two systems (i.e., the system with $c$ servers and the system with one fast server) is the same, i.e., $c\mu$, where $\mu$ is the rate of one server.
 For the system with the fast server, the utilization is defined as $\rho =\lambda/c\mu$, and for stability we require $\rho<1$.
 Another way to see it is to assume that the stream of jobs is split into $c$ smaller streams, each with arrival rate $\lambda/c$.
 In this case, applying the condition that $(\lambda/c )/\mu<1$ per server leads to the same condition that $\lambda/(c\mu) < 1$.
\end{solution}
\end{exercise}



\opt{solutionfiles}{\Closesolutionfile{hint}
\Closesolutionfile{ans}
\loadgeometry{normal}
\input{hint}
\input{ans}
}

\end{document}



%%% Local Variables:
%%% mode: latex
%%% TeX-master: t
%%% End:
