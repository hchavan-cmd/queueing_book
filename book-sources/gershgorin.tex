% arara: pdflatex: { shell: yes, interaction: nonstopmode }
% arara: pythontex: {verbose: yes, rerun: modified }
% arara: pdflatex: { shell: yes, interaction: nonstopmode }
% arara: clean: { extensions: [ aux, blg, idx, ilg, ind, log, out, pytxcode, rel, toc ] }
% !arara: clean: { files: [ ans.tex, hint.tex] }

% arara: pdflatex
% arara: clean: { extensions: [ aux, blg, idx, ilg, ind, log, out, pytxcode, rel, toc ] }
% !arara: clean: { files: [ ans.tex, hint.tex] }


\documentclass[queueing-book-solution.tex]{subfiles}
%\externaldocument{queueing-book}

\opt{solutionfiles,check}{
\loadgeometry{tufte}
\Opensolutionfile{hint}
\Opensolutionfile{ans}
}

\begin{document}



\section{On $\lambda = \gamma + \lambda P$}
\label{sec:lambda-=-gamma}

There is a remarkable similarity between~\cref{eq:93} and~\cref{eq:101}.
To see this, write the former in matrix form as $v = Pv + h$, 
with \begin{align*}
  P &=
  \begin{pmatrix}
    0 & 0 & 0 & 0&  \hdots\\
    \beta & 0  & \alpha & 0 & \hdots \\
    0 & \beta & 0 & \alpha  & 0 \\
    \vdots &  & \ddots & \ddots& \ddots
  \end{pmatrix},
& h =
                   \begin{pmatrix}
                     0 \\
                     1/(\lambda + \mu)\\
                     2/(\lambda + \mu)\\
                     \vdots
                   \end{pmatrix},
\end{align*}
and the latter as $\lambda = \gamma + \lambda P$.\sidenote{A similar form can be found for~\cref{eq:98}.}
Clearly, the former is just the transpose of the latter.
In this final section we concentrate on finding a general condition to ensure that these equations have a solution. 

We remark that, formally speaking, the matrix $P$ in $v=Pv +h$ is an infinite matrix.
Rather than dealing with the ensuing technical complications, we bypass them by restricting the analysis to an $M/M/1/K$ queue under an $N$-policy, with $K\gg N$ for any $N$ that can be `reasonable'.\sidenote{What is reasonable depends on the context of course, but $N=1000$ seems ridiculously large for any practical queueing system.}

For ease of writing, we only regard $\lambda = \gamma + \lambda P$.

\newthought{To start,} define iteratively, $\lambda^0 = 0$, $\lambda^n = \gamma + \lambda^{n-1}P$, for $n\geq 1$.
Then, by substitution, we find
\begin{align*}
  \lambda^n = \gamma + \lambda^{n-1} P = \gamma + (\gamma + \lambda^{n-2}P) P = \cdots = \gamma \sum_{m=0}^{n-1} P^m,
\end{align*}
where $P^0$ as the identity matrix.
Since $P$ is a non-negative matrix, $P^n$ is also non-negative for any $n$, hence, $\sum_{m=0}^n P^m$ is a monotone increasing sequence (of matrices).\sidenote{Formally, non-decreasing sequence.}
Thus, for $\lambda=\gamma + \lambda P$ to have a (unique) solution, this sum must remain finite as $n\to \infty$.

Now, observe the similarity with $\sum_{i=0}^n x^i$ with $|x| < 1$.
From~\cref{eq:61} we see that this converges to $1/(1-x)$ for $n\to \infty$.
By the same token, if there is an $\epsilon>0$ such that for all $i, j$, $0\leq P^m_{ij} \leq (1-\epsilon)^m$, then $0\leq \sum_{m=0}^n P_{ij}^m \leq \sum_{m=0}^n (1-\epsilon)^m \leq 1/\epsilon$.

It remains to find a condition to ensure that such $\epsilon$ exists.  For this, we discuss two ways.

\newthought{From~\cref{ex:on:3} we know} that $P_{i0} = 1-\sum_{j=1}^M P_{ij}$ is the probability that a job leaves the network from station $i$.
Next, consider a station $k$ with $P_{ki} > 0$.
Then the probability that a job starts at $k$ and leaves the network after a visit to $i$ is  $P_{ki}P_{i0}>0$.
Consequently, $P^2_{k0} = \sum_{j=0}^M P_{kj}P_{j0} \geq P_{ki}P_{i0} > 0$.


More generally, we say that $M\times M$ matrix $P$ is \recall{transient} when it is possible to leave the network from any station in at most $M$ steps.
In other words, for any station $j$ there is a sequence of intermediate stations $j_1, j_2, \ldots , j_{M-1}$ such that $P^{M}_{j0} \geq P_{j j_1}P_{j_1 j_2}\cdots P_{j_{M-1}0} > 0$.
Using this, it is not hard to prove that $P^n \leq  (1-\epsilon)^n$ when $P$ is transient.\sidenote{\cref{ex:g-3}}

\newthought{For the second} method, let us make the simplifying assumption that $P$ is a diagonalizable matrix with $M$ different eigenvalues.\sidenote{The argument below applies just as well matrices reduced to Jordan normal form, but adds only notational clutter.}
In this case, there exists an invertible matrix $V$ with the (left) eigenvectors of $P$ as its rows and a diagonal matrix $\Lambda$ with the eigenvalues of $P$ on its diagonal such that $ V P = \Lambda V$.
Hence, premultiplying with $V^{-1}$, $ P = V^{-1}\Lambda V$.
But then $P^2 = V^{-1}\Lambda V \cdot V^{-1}\Lambda V= V^{-1}\Lambda^2 V$, and in general $P^n = V^{-1}\Lambda^n V$.
Clearly, if each eigenvalue $\lambda_i$ is such that its modulus $|\lambda_i| < 1$, then $\Lambda^n \to 0$ exponentially fast, hence $P^n\to 0$ exponentially fast.

So, let us prove that all eigenvalues of a finite, transient routing matrix $P$ have modulus less than $1$.
For this we use \recall{Gerschgorin's disk theorem}.
Define the Gerschgorin disk of the $i$th row of the matrix $P$ as the disk in the complex plan:
\begin{equation*}
B_i=\left\{z\in \mathbb{C};  |z-p_{ii}|\leq \sum_{j\neq i} |p_{i j}| \right\}.  
\end{equation*}
In words, this is the set of complex numbers that lies within a distance $\sum_{j\neq i} |p_{i j}|$ of the point $p_{i i}$.
Next, assume for notational simplicity that for each row $i$ of $P$ we have that $\sum_{j} p_{i j}<1$.\sidenote{Otherwise apply the argument to $P^M$.}
Then this implies for all $i$ that
\begin{equation*}
1> \sum_{j=1}^M p_{i j} =  p_{ii} + \sum_{j\neq i} p_{i j}.
\end{equation*}
Since all elements of $P$ are non-negative, so that $|p_{i j}| = p_{i j }$, it follows that
\begin{equation*}
-1 < p_{ii} - \sum_{j\neq i} p_{i j} \leq p_{ii} + \sum_{j\neq i} p_{i j} < 1. 
\end{equation*}
With this and using that $p_{ii}$ is a real number (so that it lies on the real number axis) it follows that the disk $B_i $ lies strictly within the complex unit circle $\{z \in \mathbb{C}; |z|\leq 1\}$.
As this applies to any row $i$, the union of the disks $\cup_{i} B_{i}$ also lies strictly within the complex unit circle.
Now we invoke Gerschgorin's theorem, which states that all eigenvalues of the matrix $P$ must lie in $\cup_i B_i$, to conclude that all eigenvalues of $P$ lie strictly in the unit circle, hence all eigenvalues have modulus smaller than 1.


\newthought{Here we finish} our discussion of queueing systems, but there are many other interesting extensions to learn, for which we refer to the following work.
\begin{itemize}
\item You can find really nice discussions of networks of $M/M/\infty$, chemical reactions, population dynamics and Petri nets in \cite{baez2012quantum}, which is freely available on arXiv.
\item Simple queueing networks (networks that satisfy so-called local balance) can be modeled as electrical networks.
  For this, see \cite{doyle84:_random_walks_elect_networ}, which you can download for free from the homepage of Doyle.
\item In more general terms, queueing systems or networks are examples of Markov processes.
  A particularly nice book on these topics is \cite{norris97:_markov_chain}.
  The material of this chapter can be couched in the theory of martingales and optimal stopping.
  Besides that this is nice theory, this is widely used in quantitative finance.
\end{itemize}


\begin{exercise}
What is the  routing matrix~$P$ for a tandem network with $M$ stations? Show that $P^M = 0$.
\begin{hint}
After visiting $M$ stations, any job must have left the tandem network.
\end{hint}
\begin{solution}
$P_{i, i+1} = 1$, and $P_{i j} = 0$ for $j\neq i+1$. As $P$ is an upper-triangular matrix of dimension $M\times M$, $P^M =0$. 
\end{solution}
\end{exercise}



\begin{exercise}\label{ex:g-3}
Show that when $P$ is transient, $P^n \leq (1-\epsilon)^n$ for some $\epsilon > 0$.
\begin{hint}
  Define $Q= P^M$. As $P$ is transient, $Q_{ij} < 1$ for all $i, j$. Then show that $Q^n \to 0$. 
\end{hint}
\begin{solution}
  $P$ transient $\implies Q_{i0} = P_{i0}^M > 0 \implies \sum_{j=1}^M Q_{ij} < 1$.
  Since $M$ is a finite number, there exists an $\epsilon>0$ such that $\max\{\sum_{j=1}^M Q_{ij}\} < 1-\epsilon$.
  Writing $\bf 1$ for the vector $(1,1,\ldots, 1)$, this  means that $Q{\bf 1} < (1-\epsilon) {\bf 1}$.
  But then, $ Q^{2} {\bf 1} = Q(Q{\bf 1}) < (1-\epsilon) Q{\bf 1} < (1-\epsilon)^2 {\bf 1}$.
As $Q\geq 0$ element wise, we see that $0\leq Q^n < (1-\epsilon)^n$. And then $P^n = Q^{n/M} < (1-\epsilon)^{n/M}$. 
\end{solution}
\end{exercise}




\begin{exercise}
  Why does the assumption of \cref{ex:g-3} does not apply to an infinite transient matrix $P$?
\begin{solution}
  The above argumentation is not necessarily valid for matrices $P$ that are infinite, since $\inf\{P^{M}_{ik}\}$ need not be strictly positive for all $i, j$. 
\end{solution}
\end{exercise}

\begin{exercise}
Show that the matrix $P$  used in \cref{sec:n-policies} is transient for a finite system.
\begin{solution}
The probability to move from any state $n$ straightaway to $0$ is $\beta^n >0$. (Why do we require $n$ to be finite?)
\end{solution}
\end{exercise}

\opt{solutionfiles}{\Closesolutionfile{hint}
\Closesolutionfile{ans}
\loadgeometry{normal}
\input{hint}
\input{ans}
}

\end{document}


%%% Local Variables:
%%% mode: latex
%%% TeX-master: t
%%% End:
