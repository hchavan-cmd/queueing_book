% arara: pdflatex: { shell: yes, interaction: nonstopmode }
% arara: pythontex: {verbose: yes, rerun: modified }
% arara: pdflatex: { shell: yes, interaction: nonstopmode }
% arara: clean: { extensions: [ aux, blg, idx, ilg, ind, log, out, pytxcode, rel, toc ] }
% !arara: clean: { files: [ ans.tex, hint.tex] }

% arara: pdflatex
% arara: clean: { extensions: [ aux, blg, idx, ilg, ind, log, out, pytxcode, rel, toc ] }
% !arara: clean: { files: [ ans.tex, hint.tex] }


\documentclass[queueing-book-solution.tex]{subfiles}
%\externaldocument{queueing-book}

\opt{solutionfiles,check}{
\loadgeometry{tufte}
\Opensolutionfile{hint}
\Opensolutionfile{ans}
}

\begin{document}


\section{$G/G/c$ Queue: Sakasegawa's Formula}\label{sec:gg1}



In this section, we discuss Sakasegawa's formula by which we can estimate the expected waiting time in queue for the $G/G/c$ queue.
In the exercises we  show how to use this formula, in~\cref{sec:mg1} we provide the theoretical underpinning.




While there is no expression available to compute the exact expected waiting time for the $G/G/c$ queue, Sakasegawa's formula provides a reasonable approximation.
This takes the form
\begin{subequations}
\begin{equation}\label{eq:7}
 \E{\W} = \frac{C_a^2+C_s^2}2 \frac{\rho^{\sqrt{2(c+1)}-1}}{1-\rho} \frac{\E S}{c},
\end{equation}
where $C_a^2 = \V X/(\E X )^2$ is the SCV of the inter-arrival times, $C_s^2=\V S/(\E S)^2$ is the SCV of the service times, and
 the \emph{utilization} of the station\sidenote{Here is a subtle point. When there are multiple machines, the utilization of each machine need not be equal to $\rho$. For instance, if there is a preference to choose the `left-most' machine whenever it is free, then the utilization of this machine is larger than $\rho$. Only when the machines have the same speed, and the \emph{routing} is such that each machine receives a fraction $\lambda/c$ of jobs, the machines have the same utilization.}
is given by
\begin{equation}\label{eq:6}
 \rho = \frac{\lambda \E S}{c}
\end{equation}
\end{subequations}
where $\lambda$ is the rate at which jobs arrive at the system, $\E S$ is the expected service time, and $c$ the number of servers.




\newthought{It is crucial} to memorize the insights into the performance of queueing systems that this formula offers.
Even though~\cref{eq:7} is an approximation, it proves to be exceedingly useful when designing queueing systems and analyzing the effect of certain changes.

First, we see that $\E{\W} \sim (1-\rho)^{-1}$.
Consequently, when $\rho$ is large, the waiting time is (very) large. And,
not only is $\E\W$ large, it is also extremely \emph{sensitive} to the actual value of $\rho$.
Clearly, such situations must be avoided, and therefore, when trying to improve a queueing system, the first focus should be on reducing~$\rho$.

Second, $\rho = \lambda \E S/c$.
Thus, when $\rho$ must be made smaller, we have only three options.\sidenote{And not more!}
We can reduce the arrival rate $\lambda$ of jobs, for instance by blocking demand, or sending it elsewhere such as to another machine.\sidenote{And if customers have a choice, they will take their own measures, simply by going elsewhere.}
We can make $\E S$ smaller by replacing a server with a faster one or by shifting some of the processing steps of a job to other servers, thereby making job sizes smaller.
Finally, we can add servers, which reduce $\rho$ quite significantly when~$c$ is small.\sidenote{This is precisely what you see in supermarkets.}
If technically possible, adapting $c$ is a very effective mechanism to control waiting times.

Third, $\E{\W} \sim C_a^2$ and $\E{\W} \sim C_s^2$, which implies that when job inter-arrival or service times are very variable, $\E\W$ is large.
Thus, after ensuring that~$\rho$ is sufficiently small, it becomes important to concentrate on reducing on $C_a^2$ and $C_s^2$.\sidenote{It works the other way too: systems with low variability can deal with higher load.
}

Finally, $\E{\W} \sim \E S/c$.
This says that, from the perspective of a job in queue, average job service times are $c$ times as short as `its own service time'.


\newthought{Clearly, Sakasegawa's equation} requires an estimate of $C_a^2$ and $C_s^2$.
Now it is not always easy in practice to determine the actual service time distribution, one reason being that service times are often only estimated by a planner, but not actually measured.
Similarly, the actual arrival moments of jobs are often not registered, only just the date, or perhaps the hour, that a customer arrived.
Hence, it is often not possible to directly estimate $C_a^2$ and $C_s^2$ from the information that is available.

However, when the number of arrivals per period $\{a_n\}$ has been logged for some time, we can use $\{a_n\}$ to estimate~$C_a^2$
as\sidenote{This can of course also be used to estimate $C_s^2$.}
\begin{equation*}
C_a^2 \approx \frac{\tilde \sigma^2}{\tilde \lambda},
\end{equation*}
where
\begin{equation*}
\tilde \lambda = \lim_{n\to\infty} \frac 1n \sum_{i=1}^n a_i,\quad
\tilde \sigma^2 = \lim_{n\to\infty} \frac 1 n \sum_{i=1}^n a_i^2 - \tilde \lambda^2.
\end{equation*}
We derive this results in steps.\sidenote{It is based an argument in~\cite{cox62:_renew_theor}.}

First recall some results of earlier sections.
Let the inter-arrival time $X$ have mean $1/\lambda$ and variance $\sigma^2$, so that
\begin{equation*}
C_a^2 = \frac{\V{X_i}}{(\E{X_i})^2} = \frac{\sigma^2}{1/\lambda^2} = \lambda^2 \sigma^2.
\end{equation*}
Let $A_k$ be the arrival time of the $k$th arrival, see~\cref{eq:29}, and $A(t)$ the number of arrivals up to time $t$, see~\cref{eq:2}.
Consider the following useful relation between $A(t)$ and $A_k$, see~\cref{ex:54},
\begin{equation*}
\P{A(t) < k} = \P{A_k > t}.
\end{equation*}



Since the inter-arrival times have finite mean and second moment by assumption, we use~\cref{ex:l-147} to see that
\begin{equation*}
\lim_{k\to\infty}\frac{A_k -k/\lambda}{\sigma \sqrt k} = N(0,1),
\end{equation*}
where $N(0,1)$ is a standard normal random variable with
distribution $\Phi(\cdot)$. Similarly,
for an $\alpha$  yet to be determined,\sidenote{This is a common trick: suppose that there exists a constant to take of the scaling, and then try an extra relation to `ferret out' the scaling constant.}
\begin{equation*}
\frac{A(t) -\lambda t}{\alpha \sqrt t} \to N(0,1).
\end{equation*}
Thus,
$\E{A(t)} = \lambda t$ and $\V{A(t)}=\alpha^2 t$.

Using that $\P{N(0,1) \leq y} = \P{N(0,1) > -y}$, we have that
\begin{align*}
\Phi(y) &\approx \P{\frac{A_k - k/\lambda}{\sigma \sqrt k }\leq y} \\
 &= \P{\frac{A_k - k/\lambda}{\sigma \sqrt k} > -y} \\
 &= \P{A_k > \frac k\lambda - y \sigma \sqrt k}.
\end{align*}


We can use this relation between the distributions of~$A(t)$
and~$A_k$ to see that~$\P{A_k >t_k } = \P{A(t_k) <k}$ where
we define for ease
\begin{equation*}
t_k = \frac{k}\lambda - y \sigma \sqrt k.
\end{equation*}
With this we get,
\begin{align*}
\Phi(y)
 &\approx \P{A_k > t_k } \\
 &= \P{A(t_k) < k } \\
 &= \P{\frac{A(t_k) - \lambda t_k}{\alpha \sqrt{t_k}} <
\frac{k-\lambda t_k}{\alpha \sqrt{t_k}}}.
\end{align*}
Since $(A(t_k) - \lambda t_k)/ \alpha \sqrt{t_k} \to N(0,1)$
as $t_k \to \infty$, the above implies that
\begin{equation*}
\frac{k-\lambda t_k}{\alpha \sqrt{t_k}} \to y,
\end{equation*}
as $t_k \to \infty$. Using the above definition of $t_k$, the LHS
of this equation can be written as
\begin{equation*}
\frac{k-\lambda t_k}{\alpha \sqrt{t_k}} =
\frac{\lambda \sigma \sqrt k }{\alpha \sqrt{k/\lambda + \sigma\sqrt k}} y.
\end{equation*}
Since $t_k \to \infty$ is implied by (and implies)
$k\to\infty$, we therefore want that $\alpha$ is such that
\begin{equation*}
\frac{\lambda \sigma \sqrt k }{\alpha \sqrt{k/\lambda + \sigma\sqrt k}} y \to y,
\end{equation*}
as $k\to\infty$. This is precisely the case when
\begin{equation*}
\alpha = \lambda^{3/2}\sigma.
\end{equation*}
Finally, for $t$ large (or, by the same token $k$ large),
\begin{equation*}
\frac{\sigma_k^2}{\lambda_k} = \frac{\V{A(t)}}{\E{A(t)}} \approx \frac{\alpha^2 t}{\lambda t}
= \frac{\alpha^2}{\lambda} = \frac{\lambda^3\sigma^2}{\lambda} = \lambda^2 \sigma^2 = C_a^2,
\end{equation*}
where the last equation follows from the above definition of $C_a^2$.

\begin{exercise}
In a manufacturing setting, the Poisson process is not always a suitable model for the arrival process of jobs at a production station. Can you provide an example to see why this is the case?
\begin{hint}
Think about the inter-departure time of a machine that produces items every 10 minutes. What inter-arrival times will a down-stream machine see?

\end{hint}
\begin{solution}
  When processing times at a station are nearly constant, and the jobs of this station are sent to a second station for further processing, the inter-arrival times at the second station must be roughly equal.
  But then the inter-arrival times are not well approximated by the exponential distribution, consequently, the arrival process is not well described by a Poisson process.
\end{solution}
\end{exercise}


\begin{exercise}\label{ex:77}
 Consider \marginpar{This is an important example to memorize.} a single-server queue at which every minute a customer arrives, precisely at the first second and $S\equiv 50$ s.
 What are $\rho$, $\E\L$, $C_a^2$, and $C_s^2$?
\begin{solution}
 $\rho = \lambda \E S = 1/60 \cdot 50 = 5/6$.
 Since job arrivals do not overlap any job service, the number of jobs in the system is 1 for $50$ seconds, then the server is idle for 10 seconds, and so on.
 Thus $\E\L = 1\cdot5/6 = 5/6$.
 There is no variance in the inter-arrival times, and also not in the service times, thus $C_a^2 = C_s^2 = 0$.
 Also $\E{\W}=0$ since $\E{\Q}=0$.
\end{solution}
\end{exercise}

\begin{exercise}\label{ex:76}
 Consider the same single-server system as in~\cref{ex:77}, but now the customer service time is stochastic: with probability $1/2$ a customer requires $1$ minute and $20$ seconds of service, and with probability $1/2$ the customer requires only $20$ seconds of service.
 What are $\rho$, $C_a^2$, and $C_s^2$?
\begin{solution}
 Again $\E S$ is 50 seconds, so that $\rho = 5/6$. Also
 $C_a^2=0$. For the $C_s^2$ we have to do some work.
 \begin{equation*}
 \begin{split}
 \E S &= \frac{20}2 + \frac{80}2 = 50 \\
 \E{S^2} &= \frac{400}2 + \frac{6400} 2 = 3400 \\
 \V S &= \E S^2 - (\E S)^2 = 3400 - 2500 = 900 \\
 C_s^2 &= \frac{\V S}{(\E S)^2} = \frac{900}{2500}=\frac{9}{25}.
 \end{split}
 \end{equation*}
\end{solution}
\end{exercise}


\begin{exercise}\label{ex:91}
  (Hall 5.19) When a bus reaches the end of its line, it undergoes a series of inspections.
  The entire inspection takes 5 minutes on average, with a standard deviation of 2 minutes.
  Buses arrive with inter-arrival times uniformly distributed on $[3,9]$ minutes, hence, 10 buses arrive per hour  on average.  There is one mechanic available for the inspection.
  Use~\cref{eq:7} to  estimate $\E{\W}$.

Next, assuming that buses arrive as a Poisson process, estimate $\E\W$.

 Why is the queueing time smaller in the first setting?
\begin{hint}
  What is $\lambda$? What is $C_a^2$ in the $G/G/1$ setting; what is it in the $M/G/1$ setting?
\end{hint}

\begin{solution}
First the $G/G/1$ case. Observe that in this case, the inter-arrival time $X\sim U[3,9]$, that is, never smaller than 3 minutes, and never longer than 9 minutes.

\begin{pyconsole}
a = 3.0
b = 9.0
EX = (b + a) / 2.0  # expected inter-arrival time
EX
labda = 1.0 / EX  # per minute
labda
VA = (b - a) * (b - a) / 12.0
CA2 = VA / (EX * EX)
CA2

ES = 5.0
sigma = 2
VS = sigma * sigma
CS2 = VS / (ES * ES)
CS2

rho = labda * ES
rho

W = (CA2 + CS2) / 2.0 * rho / (1.0 - rho) * ES
W
\end{pyconsole}

Now the $M/G/1$ case. In that case $C_a^2=1$.
\begin{pyconsole}
W = (1.0 + CS2) / 2.0 * rho / (1.0 - rho) * ES
W
\end{pyconsole}

The arrival process with uniform inter-arrival times is much more regular than a Poisson process.
In the first case, bus arrivals are spaced in time at least with 3 minutes.
\end{solution}
\end{exercise}


\begin{exercise}\label{ex:l-185}
  Show for the $G/G/c/K$ queue that $\beta = 1 - \E{\Ls}\mu/ \lambda$, where $\mu=1/\E S$ is the service rate, $\beta$ the long-run fraction of customers lost, and $\E{\Ls}$ the average number of busy/occupied servers.
\begin{hint}
  Why is $\lambda \beta$ the rate at which jobs are lost?
  What, then, is the rate at which jobs are accepted to the system?
  Observe that  $\mu \E{\Ls}$ is the rate at which jobs leave the system.
\end{hint}

\begin{solution}
  Observe that $\lambda(1-\beta)$ is the net arrival rate, as jobs are lost at a rate $\lambda\beta$.
  The rate at which the station carries out work is $\mu \E{\Ls}$.
  Since all jobs that enter the system, must also leave the system (recall, the queue is finite), it follows that $\mu \E{\Ls} = \lambda(1-\beta)$, from which the expression for $\beta$ readily follows.



\end{solution}
\end{exercise}



\begin{exercise}\label{ex:49}
  A machine serves two types of jobs.
  The processing time of jobs of type $i$, $i=1,2$, is exponentially distributed with parameter $\mu_i$.
  The type $T$ of a job is random and independent of anything else, and such that $\P{T=1} = p = 1-q = 1-\P{T=2}$.
  (An example is a desk serving men and women, both requiring different average service times, and $p$ is the probability that the customer in service is a man.)
  Show that the expected processing time and variance are given by\sidenote{Thus, even if $\V{S_1} = \V{S_2} = 0$, still $\V S > 0$ if $\E{S_1} \neq \E{S_2}$.
    Mixing different jobs types increases variability, hence queueing times.}
\begin{align*}
 \E S &= p \E{S_1} + q \E{S_2} \\
\V S &= p \V{S_1} + q \V{S_2} + pq(\E{S_1} - \E{S_2})^2.
 \end{align*}


\begin{hint}
 Let $S$ be the processing (or service) time at the server, and
 $S_i$ the service time of a type $i$ job. Then,
 \begin{equation*}
 S = \1{T=1} S_1 + \1{T=2} S_2.
 \end{equation*}
\end{hint}
\begin{solution}
With the hint,
\begin{align*}
 \E S
&= \E{\1{T=1} S_1} + \E{\1{T=2} S_2} \\
&= \E{\1{T=1}} \E{S_1} + \E{\1{T=2}} \E{S_2}, \quad \text{by the independence of $T$}, \\
&= \P{T=1}\E{S_1} + \P{T=2} \E{S_2} \\
&= p \E{S_1} + q \E{S_2}.
\end{align*}

For the variance, we need some algebra. Since,
\begin{equation*}
\1{T=1}\1{T=2} = 0 \text{ and } \1{T=1}^2 = \1{T=1},
\end{equation*}
we get
\begin{align*}
 \V S
&= \E{S^2} - (\E S)^2 \\
&= \E{\left(\1{T=1} S_1 + \1{T=2} S_2\right)^2} - \left(\E{S}\right)^2 \\
&= \E{\1{T=1} S_1^2 + \1{T=2} S_2^2} - \left(\E S \right)^2 \\
&= p \E{S_1^2} + q \E{S_2^2} - \left(\E S \right)^2 \\
&= p \V{S_1} + p (\E{S_1})^2 + q \V{S_2} + q(\E{S_2})^2 - \left(\E S\right)^2 \\
&= p \V{S_1} + p (\E{S_1})^2 + q \V{S_2} + q(\E{S_2})^2 - p^2 (\E{S_1})^2 - q^2 (\E{S_2})^2 - 2pq \E{S_1}\E{S_2} \\
&= p \V{S_1} + q \V{S_2} + pq (\E{S_1})^2 +pq (\E{S_2})^2 - 2pq \E{S_1}\E{S_2}, \quad\text{ as } p = 1-q\\
&= p \V{S_1} + q \V{S_2} + pq(\E{S_1} - \E{S_2})^2.
\end{align*}
\end{solution}
\end{exercise}

\opt{solutionfiles}{\Closesolutionfile{hint}
\Closesolutionfile{ans}
\loadgeometry{normal}
\input{hint}
\input{ans}
}

\end{document}



%%% Local Variables:
%%% mode: latex
%%% TeX-master: t
%%% End:
