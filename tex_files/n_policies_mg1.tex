% arara: pdflatex
% arara: clean: { extensions: [ aux, blg, idx, ilg, ind, log, out, pytxcode, rel, toc ] }
% !arara: clean: { files: [ ans.tex, hint.tex] }

% arara: pdflatex
% arara: clean: { extensions: [ aux, blg, idx, ilg, ind, log, out, pytxcode, rel, toc ] }
% !arara: clean: { files: [ ans.tex, hint.tex] }


\documentclass[queueing-book-solution.tex]{subfiles}
%\externaldocument{queueing-book}

\opt{solutionfiles,check}{
\loadgeometry{tufte}
\Opensolutionfile{hint}
\Opensolutionfile{ans}
}

\begin{document}


\section{N-policies for the $M/G/1$ queue}
\label{sec:n-policies-mg1}

Interestingly, we can extend the analysis of~\cref{sec:distr-queue-length} and~\cref{sec:n-policies} to compute the average cost of the $M/G/1$ queue under an $N$-policy.
At the end we will find the threshold $N^*$ that minimizes the long-run average cost under the $N$-policy.

Throughout we consider the $M/G/1$ queueing process at moments at which services start.
This is similar to~\cref{sec:distr-queue-length}: When the system is not empty, these are departure epochs.


\newthought{It is easy} to obtain an expression for the clearing time $T(q)$ when a job starts with $q$ jobs in the system.
As in~\cref{sec:distr-queue-length}, write $Y$ for the number of jobs that arrive during a service time.
Then, analogous to~\cref{eq:92}, $T(q)$ satisfies the relation
\begin{equation}\label{eq:33}
  T(q) = \E S + \E{T(q+Y-1)},
\end{equation}
because first the job in service must leave, and then,  when $Y=k$,  it takes $T(q+k-1)$ to hit level~$0$.

Again, we guess that $T(q) = a q + b$.
But $b=0$ since $T(0)=0$.
Substitution of $a q$ into~\cref{eq:33} gives $q a =\E S + aq + a\E Y - a$. And noting that $\E Y = \lambda \E S,$\sidenote{\cref{ex:f-3}} we  obtain
\begin{equation*}
a = \frac{\E S}{1-\lambda \E S} \implies T(q) = \frac{\E S}{1-\lambda \E S}q.
\end{equation*}

\newthought{To find the} cost to clear the queue, we first concentrate on the expected queueing cost $U_{q}(s)$ that accrue when $q$ jobs are present and the remaining service time is~$s$.
Again we can develop a recursion by combining the ideas underlying the derivation of~\cref{eq:99} with \cref{eq:32}.

If during an interval of length $\delta$, $0<\delta\ll 1$, no new jobs arrive, the expected amount paid is $qh\delta +U_{q}(s-\delta)$.
If one job arrives, the total amount must be $(q+1)h\delta + U_{q+1}(s-\delta)$, and so on.
With probability $1-\lambda \delta$ no job arrives, with probability $\lambda \delta$ one job arrives, and since $\delta\ll 1$, the rest of the terms are $o(\delta)$.
Therefore,
\begin{align*}
  U_{q}(s)
  &=  (1-\lambda \delta)\left(qh\delta + U_{q}(s-\delta)\right) \\
  &\quad + \lambda \delta \left((q+1) h \delta + U_{q+1}(s-\delta)\right) + o(\delta) \\
  &=  U_{q}(s-\delta) + qh\delta + \lambda \delta (U_{q+1}(s-\delta) - U_{q}(s-\delta)) + o(\delta).
\end{align*}
By subtracting $U_{q}(s-\delta)$ from both sides, dividing by $\delta$, and taking the limit $\delta \to 0$, this becomes
the differential equation
\begin{align*}
  U_{q}'(s)  &=   qh + \lambda  (U_{q+1}(s) - U_{q}(s)) = q h + \lambda s,
\end{align*}
where the second equality follows from the fact that $U_{q+1}(s) - U_{q}(s) = s$, because there is precisely one more job for $s$ units of time in case there are $q+1$ jobs present rather than $q$. Since $U_{q}(0) = 0$ for all $q$,
we obtain that $U_{q}(s)= hq s + \lambda h s^2/2$.
Since $U_{q}(s)$ is the expected total cost given that the service time is~$s$,
\begin{equation*}
\E{U_{q}(S)}  =  h q \E S +  \frac 12 \lambda h \E{S^2}.
\end{equation*}
is the expected cost during a job service time. Observe that there are two components: the first is the cost paid to jobs that were present at the start of the service, the second is the cost paid to jobs that arrive during the service.


\newthought{Let $V(q)$ be} the expected queueing costs to clear the system just after a service starts and  $q$ jobs are in the system.
By analogy with~\cref{eq:33}, $V(q)$ must be the solution of
\begin{equation}  \label{eq:98}
  V(q) = \E{U_{q}(S)} + \E{V(q+Y-1)}.
\end{equation}
Now note that as in~\cref{eq:93}, the cost $\E{U_{q}(S)}$ has a term linear in~$q$ and a constant term.
As before, we substitute the form $V(q) = aq^2 + bq+c$, assemble terms with the same power in $q$, and solve for the coefficients.
After some work we arrive at\sidenote{~\cref{ex:68}--\cref{ex:nm-2}}
\begin{align*}
  V(q) = \frac{h}{2}\frac{\E S}{1-\rho} q^2 + h  \frac{ 1+ \rho C_s^2}2 \frac{\E S}{(1-\rho)^2} q.
\end{align*}

\newthought{Now that we} have the expected clearing time and cost, we can compute the long-run average cost under a general $N$-policy.
The cycle time $C(N) = N/\lambda + T(N)$, and the queueing cost $W(N)$ until level $N$ is reached while the server is idle is given by~\cref{eq:99}.
Combining all this results in the long-run average cost\sidenote{~\cref{ex:n-mg3}}
\begin{equation}  \label{eq:100}
    \frac{W(N) + K + V(N)}{C(N)}
    = h \frac{1+ C_s^2}2 \frac{\rho^2}{1-\rho} + h \rho + h \frac{N-1}2 + K \frac{\lambda(1-\rho)}N.
\end{equation}
From this, the PK-formula follows directly.\sidenote{~\cref{ex:69}}

\newthought{minimizing over $N$} gives that
\begin{equation*}
  N^* \approx \sqrt{\frac{2\lambda(1-\rho)K}{h}}.
\end{equation*}

The expression for $N^*$, called the \emph{Economic Production Quantity (EPQ)}, is a famous result in inventory theory.
Specifically, it is the optimal production quantity for a production-inventory system in which a machine switches on at cost $K$ and items pay holding cost $h$ per unit time.
By taking $\E S \to 0$, hence $\rho \to 0$, we see that $N^*$ reduces to the \emph{Economic Order Quantity (EPQ)} $\sqrt{2\lambda K/h}$.





\begin{exercise}
Explain intuitively that the system rate-stable for any $N$.
\begin{solution}
  When we switch on the server, the queue `drains' at rate $\mu-\lambda>0$, with $\mu=1/\E S$.
  Consequently, no matter how large $N$, $T(N)<\infty$.
  And, whenever the system is empty, the stochastic process restarts.
  As such cycles start over and over again, and the queue length can never `escape to infinity'.
\end{solution}
\end{exercise}


\begin{exercise}
  Why does the utilization $\rho$ not depend on $N$?
\begin{hint}
 Use the argumentation that leads to~\cref{eq:86}.
\end{hint}
\begin{solution}
  The total number $A(t)$ of job that arrive during $[0,t]$ does not depend on $N$.
  Thus, in~\cref{eq:86}, $\sum_{k=1}^{A(t)}S_k$ does not depend on $N$.
  Now use rate-stability.
\end{solution}
\end{exercise}

\begin{exercise}\label{ex:68}
Simplify $a q^2 +b q = a\E{(q+Y-1)^2} + b\E{q+Y-1}+ H(q)$, and assemble powers in $q$ to obtain:
\begin{align*}
  a &= \frac h 2 \frac{\E S}{1-\E Y} = \frac{h}{2} \frac{\E S}{1- \rho}, \\
  b(1-\E Y) &= a(\E{Y^2} - 2 \E Y + 1) + \frac 12 h \lambda \E{S^2}.
\end{align*}
\begin{hint}
\begin{align*}
  a q^2 &= a q^2, \\
  b q &= 2a q \E Y - 2a q + b q + h q \E S,\\
  0 &= a \E{Y^2} - 2a \E Y + a + b \E Y - b + \frac 12 \lambda h \E{S^2}.
\end{align*}
\end{hint}
\begin{solution}
  In the hint, the first equation is superfluous.
  In the second, $bq$ cancels at both sides, by which we find $a$.
  The third now follows.
\end{solution}
\end{exercise}

\begin{exercise}\label{ex:nm-2}
Derive
\marginpar{In a sense, this is trivial, as it is just algebra, but it is hard to get the details right.}
 the expression for $V(q)$ with the previous exercise.
\begin{hint}
Use~\cref{ex:f-3} to see that $\E{Y^2} = \lambda^2 \E{S^2} + \lambda \E S$.
\end{hint}
\begin{solution}
  For $b$, using the expressions for $\E Y$ and $\E{Y^2}$,
\begin{align*}
b(1-\E Y) &= a(\E{Y^2} - 2 \E Y + 1) + \frac 12 h \lambda \E{S^2} \\
&= \frac{h \E S}{2(1-\E Y)} (\E{Y^2} - 2 \E Y + 1) + \frac 12 h \lambda \E{S^2} \\
&= \frac{h \E S}{2(1-\lambda \E S)} \left(\lambda^2 \E{S^2} + \lambda \E S - 2 \lambda \E S + 1 +  \lambda \E{S^2}\frac{1-\lambda \E S}{\E S}\right) \\
&= \frac{h \E S}{2(1-\lambda \E S)} \left(\lambda^2 \E{S^2} - \lambda \E S + 1 +  \frac{\lambda \E{S^2}}{\E S} - \lambda^2 \E{S^2}\right) \\
&= \frac{h \E S}{2(1-\lambda \E S)} \left(1+ \frac{\lambda \E{S^2}}{\E S }  - \lambda \E S \right) \\
&= \frac{h \E S}{2(1-\lambda \E S)} \left( 1+ \frac{\lambda \E{S^2}}{\E S }  - \lambda \frac{(\E S)^2}{\E S}\right) \\
&= \frac{h \E S}{2(1-\lambda \E S)} \left( 1+ \lambda \frac{\V{S}}{\E S }\right)\\
&= \frac{h \E S}{2(1-\lambda \E S)} \left( 1+ \lambda \frac{\V{S}}{(\E S)^2 } \E S\right)\\
&= \frac{h \E S}{2(1-\lambda \E S)} \left( 1+ \rho C_s^2\right).
\end{align*}
Divide now both sides by $1-\E Y$.
\end{solution}
\end{exercise}

\begin{exercise}
 Check  that $V(q)$ reduces to that of the $M/M/1$ queue.
\begin{solution}
  For $a$, multiply the numerator and denominator by $\mu=1/ \E S$.
  For $b$, multiply by $\mu^2 = 1/(\E S)^2$, use that $C_s^1=1$ because the service times are exponentially distributed, and note that
  \begin{equation*}
    \frac{1+\rho}{1-\rho} = \frac{\mu + \lambda}{\mu-\lambda}.
  \end{equation*}
\end{solution}
\end{exercise}


\begin{exercise}\label{ex:69}
Derive the PK formula from~\cref{eq:100}.
\begin{hint}
Take  $K=0$ and $N=1$, and realize that the LHS is $h\E \L$.
\end{hint}
\begin{solution}
$h\E \L = h \E\Q + h \E\Ls$. $\E \Q = \lambda \E\W$ and $\E \Ls = \lambda \E S$.
\end{solution}
\end{exercise}


\begin{exercise}\label{ex:n-mg3}
Derive~\cref{eq:100}.
\begin{solution}
  Note first that $C(N) = N(1/\lambda + \E S / (1-\rho)) = N/(\lambda(1-\rho))$. Then,
  \begin{align*}
    \frac{V(N) + K + W(N)}{C(N)}
    &= \left(aN^2 + bN + K + h N(N-1)/2 \lambda\right) \frac{\lambda(1-\rho)}N \\
    &= \frac h 2 \rho N  + \frac h 2 \frac \rho{1-\rho} (1+\rho C_s^2) + \frac h 2 (N-1)(1-\rho) + K \frac{\lambda(1-\rho)}N \\
    &= \frac h 2 \frac \rho{1-\rho} (1+\rho C_s^2) + \frac h 2 (N-1 + \rho) + K \frac{\lambda(1-\rho)}N \\
    &= \frac h 2 \frac \rho{1-\rho} (\rho + \rho C_s^2 + 1 - \rho) + \frac h 2 (N-1 + \rho) + K \frac{\lambda(1-\rho)}N \\
    &= \frac h 2 \frac{\rho^2}{1-\rho} (1+ C_s^2) +\frac h 2 \rho + \frac h 2 (N-1 + \rho) + K \frac{\lambda(1-\rho)}N \\
    &= \frac h 2 \frac{\rho^2}{1-\rho} (1+ C_s^2) + h \rho + h \frac{N-1}2 + K \frac{\lambda(1-\rho)}N.
  \end{align*}
%With the above expressions for $a$ and $b$ the result follows immediately.
\end{solution}
\end{exercise}

\begin{exercise}
Here is another way to derive the cost paid to customers that arrive during the service.
Suppose an arriving job \emph{gets paid $hs$ directly at the moment of its arrival} when $s$ units of service time remain, but it does not get paid while waiting. Let $H(s)$ be the total cost until the service is complete. Explain that
\begin{equation*}
  H(s) = H(s-\delta) + \lambda \delta s h + o(\delta).
\end{equation*}
Then show that $H'(s) = \lambda h s$, hence $H(s) = \lambda s^{2}/2$, hence $\E{H(S)} = \lambda \E{S^{2}}/2$.
\begin{solution}
  With probability $\lambda \delta$ a new job arrives, and this job receives $h s$ right away, but nothing else.
  Then $H(s)$ is the cost due from $H(s-\delta)$ onwards plus the probability of a job arriving during $\delta$ times the cost paid to this job; the rest we neglect. The rest of the steps are as in the derivation of $U_{q}(s)$.
\end{solution}
\end{exercise}

\opt{solutionfiles}{\Closesolutionfile{hint}
\Closesolutionfile{ans}
\loadgeometry{normal}
\input{hint}
\input{ans}
}

\end{document}


%%% Local Variables:
%%% mode: latex
%%% TeX-master: t
%%% End:
