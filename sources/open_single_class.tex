% arara: pdflatex: { shell: yes, interaction: nonstopmode }
% arara: pythontex: {verbose: yes, rerun: modified }
% arara: pdflatex: { shell: yes, interaction: nonstopmode }
% arara: clean: { extensions: [ aux, blg, idx, ilg, ind, log, out, pytxcode, rel, toc ] }
% !arara: clean: { files: [ ans.tex, hint.tex] }

% arara: pdflatex
% arara: clean: { extensions: [ aux, blg, idx, ilg, ind, log, out, pytxcode, rel, toc ] }
% !arara: clean: { files: [ ans.tex, hint.tex] }


\documentclass[queueing-book-solution.tex]{subfiles}
%\externaldocument{queueing-book}

\opt{solutionfiles,check}{
\loadgeometry{tufte}
\Opensolutionfile{hint}
\Opensolutionfile{ans}
}

\begin{document}


\section{Open Single-Class Product-Form Networks}
\label{sec:jackson-networks}


Up to now our analysis focused on single-station queueing systems.
However, jobs, or patients, may need several `processing' steps at different stations.
In such cases, jobs move from one queueing system to another.
In this section we analyze such queueing networks in simple settings.
We start with two stations in tandem, and then extend to general networks.
Throughout we assume that external jobs\sidenote{i.e., new jobs that have not visited any other station before.}
arrive as a Poisson processes, and that service times at all stations are exponentially distributed.
Under this condition we are able to obtain closed-form expressions for the stationary distribution of jobs at each station.\sidenote{Recall, \cref{sec:tandem-queues} only considers the expected sojourn times in tandem networks of $G/G/c$ queues, not the distribution of the number of jobs at each station in a network of $M/M/1$ queues.}




\newthought{We start with} stating the remarkable, and crucially important, result that the \emph{inter-departure times} of jobs of an $M/M/1$ queue are exponentially distributed with rate $\lambda$, just as the inter-arrival times.\sidenote{\cref{ex:63}} This property makes analysis of a \recall{tandem network} of stations, i.e., stations placed in sequence, particularly easy.

When the first station is an $M/M/1$ queue, jobs arrive as a Poisson process, but its output process is also a Poisson process.
Therefore the second station, i.e., the station immediately downstream of station 1, receives jobs as a Poisson process.
If job service times at the second station are exponentially distributed, then this station is also an $M/M/1$ station.
But then its departure process is a Poisson process in turn, and the third station receives jobs as a Poisson process, and so on.


With this insight, it follows from~\cref{eq:el} that the expected total sojourn time in a tandem network of $M$ stations is equal to
\begin{equation*}
  \E{\J} = \sum_{i=1}^{M}\E{\J_i} = \sum_{i=1}^M \frac{\E{S_i}}{1-\rho_i},
\end{equation*}
where $\E{\J_i}$ is the sojourn time at station $i$ with $\rho_i = \lambda \E{S_i}$.



\newthought{It is not difficult} to extend the above result to general networks of $M/M/1$ queues.
For this, we first need to model such networks more formally.
In particular, we assume that the probability that a job moves to station~$j$ after completing its service at station~$i$ is independent of anything else, and is given by the number $P_{i j}\in[0,1]$.\sidenote{This is called Markov routing.}
We assemble all these probabilities in a \emph{routing matrix} $P$ such that $P_{i j}$ is the element of~$P$ on the $i$th row and $j$th column.
Define $P_{i0} = 1-\sum_{j}P_{ij}$ as the probability that a job leaves the network from station $i$.
For the network to be stable, we need to require, see~\cref{sec:lambda-=-gamma},
\begin{equation}  \label{eq:34}
P_{i0}\in [0, 1] \text{ for all } i, \text{ and } P_{i0} > 0 \text{ for at least one } i.
\end{equation}

Consider station $i$, say, and assume that jobs arrive as a Poisson process with rate $\lambda_i$.
Since service times are exponentially distributed, the departure process is also Poisson with rate $\lambda_i$.
Then, after departure, jobs are sent with probability $P_{i j}$ to station~$j$.
It follows from~\cref{ex:1} that jobs sent to station~$j$ from station $i$ form a thinned Poisson process with rate $\lambda_i P_{i j}$.

Now take the perspective of station~$j$.
This station receives a merged `stream' of thinned Poisson process from all other stations.
But, by~\cref{ex:l-103}, this merged process is also a Poisson process with rate $\sum_{i=1}^{M}\lambda_i P_{ij}$.
Above we assume that external jobs arrive at station~$j$ as a Poisson process with rate~$\gamma_j$.
Merging this process with the other Poisson stream results in the Poisson process with rate $\gamma_j + \sum_{i=1}^M \lambda_i P_{i j}$.

By assumption, service times at station $j$ are exponential, hence, the departure process of this station is Poisson.
But then the thinned process resulting from routing its departing jobs to yet other stations is again Poisson, and so on.

We arrive at the intuitively clear fact that this network behaves as a set of $M/M/1$ queues. Below we will give a formal proof of this fact for two stations.


Note that we allow for external jobs arriving at any station.
Therefore, this is an \recall{open network}.
This differs from so-called \emph{closed networks}; in such networks jobs do not enter or leave the network.

\newthought{It is evident} that, when the network is stable, all jobs that enter a station eventually must leave that station.
This insight leads us to the \recall{traffic equations}, which state that for each station $i$ the departure rate must equal the arrival rate, i.e.,
\begin{equation}\label{eq:101}
  \lambda_i = \gamma_i + \sum_{j=1}^M \lambda_j P_{j i}, \quad i = 1, \ldots, M.
\end{equation}
For the \emph{network as whole}, the total external arrival rate is given by $|\gamma|= \sum_{i=1}^M \gamma_i$.
Hence, this is also the departure rate of the network.\sidenote{Not for an individual station $i$, because $\delta_i = \lambda_i$.}

\newthought{Let us for} the moment assume that we can solve the traffic equations, in other words, for given $\gamma =(\gamma_1, \ldots, \gamma_M)$ and routing matrix~$P$ we can find a set of numbers $\lambda =(\lambda_1, \ldots, \lambda_M)$  that satisfies \cref{eq:101}.
Then, we can define $\rho_i = \lambda_i \E{S_i}$ as the utilization of (the server of) station~$i$.
Clearly, we assume that $\rho_i < 1$ for all stations~$i$.

Evidently, the average total number of jobs in this network of $M/M/1$ stations is equal to the sum of the average number of jobs at each station, hence,
%\begin{equation*}
$\E\L = \sum_{i=1}^M \E{\L_i}$.
%\end{equation*}
With this we can also find an expression for the expected total sojourn time in the network $\E\J$.
Applying Little's law first to the network as a whole and then to each station individually, we see that
\begin{equation*}
  |\gamma| \E\J = \E\L = \sum_{i=1}^M \E{\L_i} = \sum_{i=1}^M \lambda_i \E{\J_i} = \sum_{i=1}^M \lambda_i \frac{\E{S_i}}{1-\rho_i}.
\end{equation*}
Dividing by $|\gamma|$ gives  $\E\J$  in terms of the \recall{visit ratios} $\lambda_i/|\gamma|$,
\begin{equation*}
 \E\J = \sum_{i=1}^M \frac{\lambda_i}{|\gamma|} \frac{\E{S_i}}{1-\rho_i}.
\end{equation*}
This is intuitive: the visit ratio $\lambda_i/|\gamma|$ tells how often station $i$ is visited relative to the total number of arrivals.
Thus, $\E{\J_i} \lambda_i /|\gamma|$ is the amount of time an `average' job spends at station~$i$ before leaving the network.

\newthought{Above we derived} expressions for the average waiting time in a network of $M/M/1$ queues, but it is possible to obtain a much stronger result.
In fact, the stationary probability $p(n)$ that the system contains $n=(n_1,n_2, \ldots, n_M)$ at stations $1,\ldots, M$ can be written as the \emph{product} of the probabilities of the individual stations, specifically,
\begin{equation*}
p(n) =  \P{\L_1=n_1, \ldots, \L_M=n_M}  = \Pi_{i=1}^M p(n_i). % = \Pi_{i=1}^M (1-\rho_i)\rho_i^{n_i},
\end{equation*}
where $p(n_i)=(1-\rho_i)\rho_i^{n_i}$ is the stationary probability that station~$i$ contains $n_i$ jobs, compare~\cref{eq:23}.
In other words, the stationary probability $p(n_i)$ of station $i$ is \emph{independent} of the state of the other stations. Hence, in stationarity, we can concentrate on each station individually. As a consequence, we can easily compute the excess probability $\P{\L_1> n_1}$ for each station $i$.

As a matter of fact, any stable open network of $M/M/c$ stations\sidenote{Note, $M/M/c$ queues, not just $M/M/1$ queues.}
admits a \recall{product-form solution}, a result known as \recall{Jackson's theorem}.

\newthought{Let us prove} this result for the case of two $M/M/1$ stations in tandem.
Let $p(i,j)=\P{\L_1=i, \L_2=j}$ be the probability that station~$1$ contains~$i$ jobs and station~$2$ contains~$j$ jobs.
Specifically, we have to show that when $p(i,j)$ has the form $(1-\rho_1)(1-\rho_2)\rho_1^i \rho_2^j$, the balance equations are satisfied for all $i, j\geq 0$, see~\cref{sec:level-cross-balance}.


Here we focus on the case $i, j \geq 1$, see~\cref{fig:tandem_transitions}.
As the normalization factor appears at both sides of the balance equations, we write $p(i,j) = \rho_1^i \rho_2^j$ for ease. Then it follows that,
Clearly, the balance equations hold for $i, j \geq 1$.
\begin{marginfigure}
\begin{tikzpicture}[scale=0.8,->,>=stealth',shorten >=1pt,auto,node distance=1.8cm]
 \node (0) {$i-1, j$} ;
 \node (1) [right of=0] {$i, j$};
 \node (2) [right of=1] {$i+1, j$};
 \node (3) [above of=1] {$i,j+1$};
 \node (4) [below of=1] {$i, j-1$};
 \node (5) [below of=2] {$i+1, j-1$};
 \node (6) [above of=0] {$i-1, j+1$};
% \node[state] (4) [right of=3] {$\cdots$};


%\draw[dashed] (3.4,-1.8) rectangle (5.6,1.8);

\path
 (1) edge  node[fill=white] {$\lambda$} (2)
 (0) edge  node[fill=white] {$\lambda$} (1)
 (1) edge  node[fill=white, above] {$\mu_1$} (6)
 (3) edge  node[fill=white] {$\mu_2$} (1)
 (5) edge  node[fill=white, above] {$\mu_1$} (1)
 (1) edge  node[fill=white] {$\mu_2$} (4);
\end{tikzpicture}
 \caption{Transitions in  two stations in tandem.}
 \label{fig:tandem_transitions}
\end{marginfigure}
\begin{align*}
    \text{rate out } &=(\lambda + \mu_1 + \mu_2) p(i, j) \\
    &= \lambda  \rho_1^i \rho_2^j + \mu_1 \rho_1^{i} \rho_2^j + \mu_2 \rho_1^i\rho_2^j\\
    &=\mu_2 \rho_1^{i} \rho_2^{j+1} + \lambda \rho_1^{i-1} \rho_2^j + \mu_1 \rho_1^{i+1}\rho_2^{j-1}\\
                     &= \mu_2p(i, j+1) + \lambda p(i-1, j) + \mu_1 p(i+1, j-1)\\
    &= \text{ rate in}.
\end{align*}
It remains to check  the boundary\sidenote{\cref{ex:72}--\cref{ex:75}} to complete the claim.




\begin{exercise}\label{ex:63}
Show that the inter-departure times $D_k -D_{k-1} \sim \Exp(\lambda)$.
\begin{hint}
Conditioning on the server being idle or busy at a departure leads to
$f_D(t) = f_{X+S}(t) \P{\text{server is idle}} + f_S(t) \P{\text{ server is busy }}$.
Next, use~\cref{ex:46}.
\end{hint}
\begin{solution}
 \begin{align*}
 f_D(t)
&= (1-\rho) f_{X+S}(t) + \rho \mu e^{-\mu t}
= (1-\rho) \frac{\mu\lambda}{\lambda-\mu} \left(e^{-\mu t}-e^{-\lambda t}\right) + \rho \mu e^{-\mu t} \\
&= \left(1-\frac{\lambda}\mu\right) \frac{\mu\lambda}{\lambda-\mu}\left(e^{-\mu t}-e^{-\lambda t}\right) + \rho \mu e^{-\mu t} \\
&= \frac{\mu-\lambda}\mu \frac{\mu\lambda}{\lambda-\mu}\left(e^{-\mu t}-e^{-\lambda t}\right) + \frac\lambda \mu \mu e^{-\mu t}
= - \lambda\left(e^{-\mu t}-e^{-\lambda t}\right) + \lambda e^{-\mu t}
 \end{align*}
\end{solution}
\end{exercise}


\begin{exercise}\label{ex:on:3}
  Why do we require~\cref{eq:34}?
\begin{hint}
 What does it mean when $P_{i 0 } > 0$?
\end{hint}
\begin{solution}
  A job leaving station $i$ has to go somewhere, to another station, perhaps to station $i$ again, or leave the network.
  As $\sum_{i=1}^MP_{i j}$ is the probability to be sent to some station in the network, $P_{i 0} $ is the probability a job leaves the network from station $i$.
\end{solution}
\end{exercise}


\begin{exercise}\label{ex:47}
We have a two-station single-server open network.
Jobs enter the network at the first station with rate $\gamma$.
A fraction $\alpha$ returns from station 1 to itself; the rest moves to station 2.
At station 2 a fraction $\beta_2$ returns to station 2 again, a fraction $\beta_1$ goes to station 1.

First, compute $\lambda$, then analyze what happens if $\alpha\to 1$ or $\beta_1\to 0$.
\begin{solution}
 \begin{align*}
 P &=
 \begin{pmatrix}
 \alpha & 1- \alpha \\
 \beta_1 & \beta_2
 \end{pmatrix},
&
 (\lambda_1, \lambda_2) &= (\gamma, 0) + (\lambda_1, \lambda_2) P.
 \end{align*}
 Solving first for $\lambda_2$ leads to $\lambda_2 = (1-\alpha) \lambda_1 + \beta_2 \lambda_2$, so that
\begin{equation*}
 \lambda_2 = \frac{1-\alpha}{1-\beta_2} \lambda_1.
\end{equation*}
Next, using this and that $\lambda_1 = \alpha \lambda_1 + \beta_1 \lambda_2 + \gamma$ gives
\begin{equation*}
 \begin{split}
\gamma
&= \lambda_1(1-\alpha) - \beta_1\lambda_2
= \lambda_1\left(1-\alpha - \beta_1\frac{1-\alpha}{1-\beta_2}\right) \\
&= \lambda_1(1-\alpha)\left(1 - \frac{\beta_1 }{1-\beta_2}\right)
= \lambda_1(1-\alpha)\frac{1-\beta_1-\beta_2 }{1-\beta_2}.
 \end{split}
\end{equation*}
Hence,
\begin{align*}
 \lambda_1 &= \frac\gamma{1-\alpha}\frac{1-\beta_2}{1-\beta_1-\beta_2},&
 \lambda_2 &= \frac{1-\alpha}{1-\beta_2} \lambda_1 = \frac\gamma{1-\beta_1-\beta_2}.
\end{align*}

We want of course that $\lambda_1 < \mu_1$ and $\lambda_2 < \mu_2$.
With the above expressions this leads to conditions on $\alpha$, $\beta_1$ and $\beta_2$.
Note that we have three parameters, and two equations; there is not a single condition from which the stability can be guaranteed.
If $\alpha\uparrow 1$, the arrival rate at node $1$ explodes.
If $\beta_1=0$ no jobs are sent from node 2 to node 1.
\end{solution}
\end{exercise}


\begin{exercise}\label{ex:72}
Check that $p(0,0)$ satisfy the balance equation for state $(0,0)$.
\begin{hint}
  Realize that an arrival is required to leave state $(0,0)$, and a departure at the second queue is necessary to enter state $(0,0)$.
\end{hint}
\begin{solution}
  The rate into state $(0,0)$ is $\mu_2 p(0,1) = \mu_2 \rho_2$. The rate out of state $(0,0)$ is $\lambda p(0,0) = \lambda$. Since $\rho_2=\lambda/\mu_2$, these two rates are the same.
\end{solution}
\end{exercise}

\begin{exercise}
Check that $p(i,0)$ satisfy the balance equation for state $(i,0)$.
\begin{solution}
  \begin{align*}
    \text{rate out } &=(\lambda + \mu_1) p(i,0) = \mu_1 \rho_1^i + \lambda \rho_1^i = \lambda \rho_1^{i-1} + \mu_2 \rho_1^i \rho_2 \\
    &= \lambda p(i-1,0) + \mu_2 p(i, 1) = \text{ rate in}.
  \end{align*}
\end{solution}
\end{exercise}

\begin{exercise}\label{ex:75}
Check that $p(0,j)$ satisfy the balance equation for state $(0,j)$.
\begin{solution} We show that the rate out is the rate in.
  \begin{align*}
    \text{rate out } &=(\lambda + \mu_2) p(0, j) = \mu_2 \rho_2^j + \lambda \rho_2^j
                     = \mu_1\rho_1 \rho_2^{j-1} + \mu_2 \rho_2^{j+1} \\
    &= \mu_1 p(1,j-1) + \mu_2 p(0, j+1) = \text{ rate in}.
  \end{align*}
\end{solution}
\end{exercise}

\opt{solutionfiles}{\Closesolutionfile{hint}
\Closesolutionfile{ans}
\loadgeometry{normal}
\input{hint}
\input{ans}
}

\end{document}


%%% Local Variables:
%%% mode: latex
%%% TeX-master: t
%%% End:
