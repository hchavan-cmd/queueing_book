% arara: pdflatex: { shell: yes, interaction: nonstopmode }
% arara: pythontex: {verbose: yes, rerun: modified }
% arara: pdflatex: { shell: yes, interaction: nonstopmode }
% arara: clean: { extensions: [ aux, blg, idx, ilg, ind, log, out, pytxcode, rel, toc ] }
% !arara: clean: { files: [ ans.tex, hint.tex] }

% arara: pdflatex
% arara: clean: { extensions: [ aux, blg, idx, ilg, ind, log, out, pytxcode, rel, toc ] }
% !arara: clean: { files: [ ans.tex, hint.tex] }


\documentclass[queueing-book-solution.tex]{subfiles}
%\externaldocument{queueing-book}

\opt{solutionfiles,check}{
\loadgeometry{tufte}
\Opensolutionfile{hint}
\Opensolutionfile{ans}
}

\begin{document}


\section{Server Adjustments}
\label{sec:non-preempt-interr}


In~\cref{sec:setups-batch-proc} we studied the effect of setup times between batches with constant size~$B$.
However, the server can be interrupted for other reasons than setups.
For instance, parts in a machine, such as knives, may need to be replaced or adjusted.
Such small tasks can be carried out in between jobs, but it can be hard to predict when it is required.\sidenote{For instance, the breaking of a chisel is a random event.}
Another example is a GP who  to make a unexpected phone calls between two seeing two patients.
In all such cases, the number of jobs\sidenote{You can compare this to a batch of jobs.}
served between any two adjustments\sidenote{And this to a setup.}
is no longer constant.
But we know from Sakasegawa's formula that randomness in service times affects queueing times, and we also know from~\cref{eq:60} that the batch size affects the service time.
Hence, unplanned adjustments must have an effect on sojourn times.


In this section we develop a simple model to understand the impact of such outages on job sojourn times; we use the same notation as in~\cref{sec:setups-batch-proc} and follow the same line of reasoning.
With this model, we can analyze a number of trade-offs, such doing fewer, but longer adjustments, or planning adjustments instead just waiting until it becomes necessary at an unexpected moment.\sidenote{~\cref{ex:104}} With the model we can make graphs of the sojourn time as a function of adjustment rate, for instance, so that we can optimize for the adjustment rate.

It is important to realize that setups and adjustments are both types of \recall{non-preemptive outages}, i.e., they occur \emph{between} jobs, not \emph{during}  job service times. 




We assume that run lengths $\{B_i\}$, i.e., the number of jobs served between two adjustments, forms a sequence of geometric iid\ random variables such that $\P{B=k} = (1-p)^{k-1}p$, where $p$ is the `success' probability.\sidenote{Recall from~\cref{eq:81} that the run-length $B$ is memoryless under these assumptions.}
As a consequence, $\E B=1/p$ and variance $\V B = (1-p)/p^2$.\sidenote{To obtain $p$ in practice, we measure the number of jobs served between a number of adjustments, and take the mean.
} Suppose also that the adjustment (repair) times are iid with mean $\E R$ and variance $\V R$.

To compute $\E\W$ we only have to find an expression how the adjustments affect the mean and SCV $C_s^2$ of the processing times.
All other terms in the $G/G/1$ waiting time formula remain the same, as the adjustments do not affect the job arrival process,
It is simple\sidenote{~\cref{ex:l-256}} to see that the average effective processing time is
\begin{equation} \label{eq:88}
 \E{S} = \E{S_0} + p \E R = \E{S_0} + \frac{\E R}{\E B}. 
\end{equation}
Consequently, the effective\sidenote{Hence, including down-times!}   load becomes $\lambda \E{S}$, which is equal to $\rho$ since there one server.
With this and an expression\sidenote{\cref{ex:78}} for $\E{S^2}$, we find\sidenote{\cref{ex:l-257}} after some polishing 
\begin{equation}\label{eq:89} 
  \begin{split}
 \V{S} 
&= \V{S_0} + p\V{R} + p (1-p)(\E R)^2 \\
&= \V{S_0} + \frac{\V R +(\E R)^2C_B^2}{\E B},
  \end{split}
\end{equation}
where $C_B^2$ is the SCV of the runlength $B$. We can now  fill in Sakasegewa's formula. 


\newthought{Let us compare} the results of this section to those of~\cref{sec:non-preempt-interr}.
In both cases the expected service time is the same: an amount $\E R/ \E B$ gets added to the net processing time $\E{S_0}$.
However, the impact on the variance is different.
By comparing~\cref{eq:107} to~\cref{eq:89} we see that the former is smaller.
This supports our intuition: Unexpected interruptions must have a larger effect on variability than expected interruptions.



\begin{exercise}\label{ex:l-255}
 Jobs\marginpar{We first show how to apply the model before we deal with the derivations. }
arrive as a Poisson process with rate $\lambda=9$ per working day.
 The machine works two $8$ hour shifts a day.
 Work not processed on a day is carried over to the next day.
 Job service times are 1.5 hours, on average, with standard deviation of $0.5$ hours.
 Outages occur on average between $30$ jobs.
The average duration of an outage  is $5$ hours and has a standard deviation of $2$ hours.
 Compute $\E\J$.
\begin{hint}
 Get the units right.  Compute the load, and then the rest.
\end{hint}
\begin{solution}
First we determine the load. 
\begin{pyconsole}
EB = 30
p = 1 / EB
ES0 = 1.5
labda = 9.0 / (2 * 8)  # arrival rate per hour
ER = 5.0
ES = ES0 + p * ER
ES
rho = labda * ES
rho
\end{pyconsole}
So, at least the system is stable.
\begin{pyconsole}
VS0 = 0.5 * 0.5
VR = 2.0 * 2.0
VS = VS0 + p * VR + p * (1 - p) * ER * ER
VS
Ce2 = VS / (ES * ES)
Ce2
\end{pyconsole}
And now we can fill in the waiting time formula.
\begin{pyconsole}
Ca2 = 1  # Poisson arrivals
EW = (Ca2 + Ce2) / 2 * rho / (1 - rho) * ES
EW
EJ = EW + ES
EJ
\end{pyconsole}
\end{solution}
\end{exercise}


\begin{exercise}\label{ex:104}
In the setting of~\cref{ex:l-255} we can perhaps choose to do an adjustment after every 20 jobs.
\marginpar{In maintenance this is a common problem. Should we wait until a failure occurs, or should we plan the repairs?}
For simplicity, assume that then adjustments never occur at random.
Also assume that an adjustment takes less time, for instance $4.5$ hour and are constant.
\marginpar{Because planned adjustment can be prepared.}
What is $\E\J$ now? 
\begin{hint}
Realize that we now deal with setups and batch processing.   
\end{hint}
\begin{solution}
We can use the model of~\cref{sec:setups-batch-proc}.
\begin{pyconsole}
B = 20
ER = 4.5
VR = 0
ES = ES0 + ER / B
ES
rho = labda * ES
rho
VS = VS0 + VR / B
Ce2 = VS / (ES * ES)
Ce2
EW = (Ca2 + Ce2) / 2 * rho / (1 - rho) * ES
EW
EJ = EW + ES
EJ
\end{pyconsole}
Comparing this to the results of~\cref{ex:l-255}, we see that the load becomes somewhat higher. Since $\rho$ becomes close to one, doing adjustments regularly is not a good idea.   
\end{solution}
\end{exercise}


\begin{exercise}\label{ex:l-256}
Derive \cref{eq:88}.
\begin{hint}
  If there is no interruption, the service time is $S_0$.
  However, if there is an interruption, the service of a job is $R + S_0$.
\end{hint}
\begin{solution}
 \begin{equation*}
 \E{S} = (1-p)\E{S_0} + p (\E{R} + \E{S_0}) = \E{S_0}  + p \E{R}.
 \end{equation*}
\end{solution}
\end{exercise}

\begin{exercise}\label{ex:78}
 Show that
 \begin{equation*}
 \E{S^2} = \E{S_0^2} + 2 p \E{S_0} \E R + p \E{R^2}.
 \end{equation*}
\begin{solution}
 \begin{align*}
 \E{S^2} 
&= (1-p) \E{S_0^2} + p \E{(S_0+R)^2}\\
&= (1-p) \E{S_0^2} + p \E{S_0^2} + 2 p \E{S_0} \E R + p \E{R^2} \\
&=  \E{S_0^2} + 2 p \E{S_0} \E R + p \E{R^2}. 
 \end{align*}
\end{solution}
\end{exercise}

\begin{exercise}\label{ex:l-257}
Derive \cref{eq:89}.
\begin{hint}
 First compute $\E{S^2}$. See~\cref{ex:78}.
\end{hint}
\begin{solution}
 \begin{equation*}
 \begin{split}
\V{S} 
&=\E{S^2} - (\E{S})^2 \\
&= \E{S_0^2} + 2 p \E{S_0} \E R + p \E{R^2} \\
&\quad - (\E{S_0})^2 - 2p \E{S_0}\E R - (p \E R)^2\\
&= \V{S_0} + p(\E{R^2} - (\E R)^2) + (\E R)^2p (1-p) \\
&= \V{S_0} + p\V R + (\E R)^2p (1-p) \\
&= \V{S_0} + p\V R + p^3 (\E R)^2\frac{1-p}{p^2} \\
&= \V{S_0} + p\V R + p^3 (\E R)^2 \V B\\
&= \V{S_0} + p\V R + p (\E R)^2 C^2_B.
 \end{split}
 \end{equation*}
\end{solution}
\end{exercise}

\opt{solutionfiles}{\Closesolutionfile{hint}
\Closesolutionfile{ans}
\loadgeometry{normal}
\input{hint}
\input{ans}
}

\end{document}



%%% Local Variables:
%%% mode: latex
%%% TeX-master: t
%%% End:
